\documentclass[letterpaper,DIV=18,headsepline,13pt]{scrartcl}
\usepackage[spanish,mexico,shorthands=off,es-lcroman]{babel}
\usepackage{stix2}
\pagenumbering{gobble}
%\setlength{\parindent}{0cm}

\usepackage{mathtools}
\usepackage{amsthm}
\usepackage{tikz-cd}
\usepackage{comment}
\usepackage{lipsum}
\usepackage{ifthen}
\usepackage{truthtable} %para las tablas de verdad

\usepackage{multicol}
\usepackage{paracol}
\usepackage[shortlabels]{enumitem}
\setenumerate[1]{label=\MakeLowercase{\roman*}), ref=\roman*}
\setenumerate[2]{label=\MakeLowercase{\arabic*}), ref=\alph*}

% Para escribir el "tal que" de los conjuntos
\providecommand\st{\;|\;}

%Para el uso de \Set y \Set*
\providecommand\given{}
\newcommand\SetSymbol[1][]{\nonscript\:#1\vert\allowbreak\nonscript\:\mathopen{}}
\DeclarePairedDelimiterX\Set[1]\{\}{\renewcommand\given{\SetSymbol[\delimsize]}#1}
\DeclarePairedDelimiterX\Par[1](){#1}
%Ejercicios
%\newcommand{\pts}[1]{%
  %\ifthenelse{\equal{#1}{1}}{\hfill \textbf{(#1 pt)}}{\hfill\textbf{(#1 pts)}}
%}

%Para repetir un bloque sin tener que repetir el código
%Falta mejorar, es una caja de LaTeX, por lo que no se puede romper
\ExplSyntaxOn
\box_new:N \l_repetir_caja
\NewDocumentEnvironment{Repeat}{m +b}{
  \box_clear:N \l_repetir_caja
  \vbox_set:Nn \l_repetir_caja {#2}
  \prg_replicate:nn {#1} {\box_use:N \l_repetir_caja }
}{}
\ExplSyntaxOff

\newcounter{Ejer}
\newcounter{Pts}
\setcounter{Ejer}{1}
\setcounter{Pts}{1}
\newcommand{\pts}{}
\newenvironment{ejercicio}[1]{\noindent
    \ifthenelse{\equal{#1}{1}}{\renewcommand{\pts}{\textbf{(#1 pt)}}}{\renewcommand{\pts}{\textbf{(#1 pts)}}}\textbf{Ej. \theEjer} \pts\stepcounter{Ejer}}{\vspace{.1cm}}

%Comandos que utilizamos
\newcommand{\id}{\mathrm{id}}
\newcommand{\op}{{}^{\mathrm{op}}}
\newcommand{\set}[1]{\{#1\}}
\renewcommand{\emptyset}{\varnothing}
\DeclareMathOperator{\ima}{ima}
\DeclareMathOperator{\dom}{dom}

%Entorno de Demostración y Solución
\renewcommand\qedsymbol{$\blacksquare$}
\makeatletter
    \renewenvironment{proof}[1][]{%
        \par\pushQED{\qed}%
        \normalfont\topsep6pt \partopsep0pt % espacio antes del entorno
        \trivlist
        \item[\hskip\labelsep
                \textbf{\textit{Demostración.}}% Cambiado aquí
        ]#1
        }{%
        \popQED\endtrivlist\@endpefalse
    }
\makeatother

\makeatletter
    \newenvironment{solucion}[1][]{%
        \par\pushQED{\hfill \lozenge}%
        \normalfont\topsep6pt \partopsep0pt % espacio antes del entorno
        \trivlist
        \item[\hskip\labelsep
                \textbf{\textit{Solución.}}% Cambiado aquí
        ]#1
        }{%
        \popQED\endtrivlist\@endpefalse
    }
\makeatother

\begin{document}

\begin{Repeat}{2}
    \begin{center}
        {\fontsize{30}{40}\rmfamily \textbf{Tercer Examen Parcial}} \\ \vspace{.2cm}
        Álgebra Superior 1, 2025-4
    \end{center}

    %\noindent\textit{\textbf{Instrucciones.} Resuelve los siguientes ejercicios, se pueden utilizar libremente resultados vistos en clase, siempre y cuando, se indique claramente dónde y cuáles se utilizan.}\vspace{.4cm}

    \begin{ejercicio}{4}
        Pruebe que para cualquier natural $n$ se cumple $\displaystyle \sum_{k=0}^{n} k^2 = \frac{n(n+1)(2n+1)}{6} $.
    \end{ejercicio}

    \begin{ejercicio}{3}
        Sean $A$  y $B$ conjuntos, con $B \subseteq A$. Prueba que, si $B$ es finito y $A$ infinito, entonces $A \setminus B$ es infinito.
    \end{ejercicio}

    \begin{ejercicio}{4}
        Sean $x,y \in \mathbb{R}$ y $a_{900},a_{899}, \dots ,a_1,a_0$ son los coeficientes (en orden) del polinomio $(x+y)^{900}$; es decir $ (x+y)^{900} = a_{900} x^{900} + a_{889} x^{889}y + \cdots + a_1 xy^{889} + a_0 y^{900}. $ ?`Cuál de los siguientes números es mayor, $a_{100}$ o $a_{798}$? Demuestra todas tus afirmaciones.
    \end{ejercicio}

    \begin{ejercicio}{+1}
        \textit{Este ejercicio es opcional y sólo se tomará en cuenta si no hay errores en la solución.}
        Sea $A \subseteq \mathbb{N}$ y supongamos que $\forall x \big( x \in A \to x+1 \in A \big)$. Prueba que si $m \in \mathbb{N} \cap A$, entonces $\{ n \in \mathbb{N} \mid n \geq m \} \subseteq A$.
    \end{ejercicio}
    \vspace{.5cm}

\end{Repeat}
\end{document}