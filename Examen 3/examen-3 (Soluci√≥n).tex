\documentclass[letterpaper,DIV=18,headsepline,13pt]{scrartcl}
\usepackage[spanish,mexico,shorthands=off,es-lcroman]{babel}
\usepackage{stix2}
\pagenumbering{gobble}
%\setlength{\parindent}{0cm}

\usepackage{mathtools}
\usepackage{amsthm}
\usepackage{tikz-cd}
\usepackage{comment}
\usepackage{lipsum}
\usepackage{ifthen}
\usepackage{truthtable} %para las tablas de verdad

\usepackage{multicol}
\usepackage{paracol}
\usepackage[shortlabels]{enumitem}
\setenumerate[1]{label=\MakeLowercase{\roman*}), ref=\roman*}
\setenumerate[2]{label=\MakeLowercase{\arabic*}), ref=\alph*}

% Para escribir el "tal que" de los conjuntos
\providecommand\st{\;|\;}

%Para el uso de \Set y \Set*
\providecommand\given{}
\newcommand\SetSymbol[1][]{\nonscript\:#1\vert\allowbreak\nonscript\:\mathopen{}}
\DeclarePairedDelimiterX\Set[1]\{\}{\renewcommand\given{\SetSymbol[\delimsize]}#1}
\DeclarePairedDelimiterX\Par[1](){#1}
%Ejercicios
%\newcommand{\pts}[1]{%
  %\ifthenelse{\equal{#1}{1}}{\hfill \textbf{(#1 pt)}}{\hfill\textbf{(#1 pts)}}
%}

%Para repetir un bloque sin tener que repetir el código
%Falta mejorar, es una caja de LaTeX, por lo que no se puede romper
\ExplSyntaxOn
\box_new:N \l_repetir_caja
\NewDocumentEnvironment{Repeat}{m +b}{
  \box_clear:N \l_repetir_caja
  \vbox_set:Nn \l_repetir_caja {#2}
  \prg_replicate:nn {#1} {\box_use:N \l_repetir_caja }
}{}
\ExplSyntaxOff

\newcounter{Ejer}
\newcounter{Pts}
\setcounter{Ejer}{1}
\setcounter{Pts}{1}
\newcommand{\pts}{}
\newenvironment{ejercicio}[1]{\noindent
    \ifthenelse{\equal{#1}{1}}{\renewcommand{\pts}{\textbf{(#1 pt)}}}{\renewcommand{\pts}{\textbf{(#1 pts)}}}\textbf{Ej. \theEjer} \pts\stepcounter{Ejer}}{\vspace{.3cm}}

%Comandos que utilizamos
\newcommand{\id}{\mathrm{id}}
\newcommand{\op}{{}^{\mathrm{op}}}
\newcommand{\set}[1]{\{#1\}}
\renewcommand{\emptyset}{\varnothing}
\DeclareMathOperator{\ima}{ima}
\DeclareMathOperator{\dom}{dom}

%Entorno de Demostración y Solución
\renewcommand\qedsymbol{$\blacksquare$}
\makeatletter
    \renewenvironment{proof}[1][]{%
        \par\pushQED{\qed}%
        \normalfont\topsep6pt \partopsep0pt % espacio antes del entorno
        \trivlist
        \item[\hskip\labelsep
                \textbf{\textit{Demostración.}}% Cambiado aquí
        ]#1
        }{%
        \popQED\endtrivlist\@endpefalse
    }
\makeatother

\makeatletter
    \newenvironment{solucion}[1][]{%
        \par\pushQED{\hfill \lozenge}%
        \normalfont\topsep6pt \partopsep0pt % espacio antes del entorno
        \trivlist
        \item[\hskip\labelsep
                \textbf{\textit{Solución.}}% Cambiado aquí
        ]#1
        }{%
        \popQED\endtrivlist\@endpefalse
    }
\makeatother

\begin{document}
    \begin{center}
        {\fontsize{30}{40}\rmfamily \textbf{Tercer Examen Parcial}} \\ \vspace{.2cm}
        Álgebra Superior 1, 2025-4
    \end{center}
    \begin{flushright}
        \footnotesize \hfill Profesor: Luis Jesús Trucio Cuevas.\\
        \hfill Ayudante: Hugo Víctor García Martínez.
    \end{flushright} 

    \begin{ejercicio}{4}
        Pruebe que para cualquier natural $n$ se cumple $\displaystyle \sum_{k=0}^{n} k^2 = \frac{n(n+1)(2n+1)}{6} $.
    \end{ejercicio}
    \begin{proof}
        Sea $\varphi(n)$ la propiedad:
        \[ \sum_{k=0}^{n} k^2 = \frac{n(n+1)(2n+1)}{6} \]
        Veamos por inducción que $\forall n \in \mathbb{N} \big( \varphi(n) \big)$.
        \begin{enumerate}[\hspace{1cm}]
            \item \underline{Base.} Se cumple $\varphi(0)$; efectivamente:
            \begin{align*}
                \sum_{k=0}^{0} k^2 &= 0^2 \\
                & = 0 \\
                & = \frac{0 \cdot (0+1) \cdot (2\cdot 0 + 1)}{6} 
            \end{align*}
            \item \underline{Paso inductivo.} Sea $n \in \mathbb{N}$ y supongamos que $\varphi(n)$ (H.I.), esto es:
            \[ \sum_{k=0}^{n} k^2 = \frac{n(n+1)(2n+1)}{6} \]
            Veamos que $\varphi(n+1)$, es decir:
            \[ \sum_{k=0}^{n+1} k^2 = \frac{(n+1)(n+2)(2(n+1)+1)}{6} \]
            En efecto:
            \begin{align*}
                \sum_{k=0}^{n+1} k^2 &= \sum_{k=0}^{n} k^2 + (n+1)^2 \\
                &= \frac{n(n+1)(2n+1)}{6} + (n+1)^2 \tag{por H.I.} \\
                &= \frac{n(n+1)(2n+1) + 6(n+1)^2}{6} \\
                &= \frac{(n+1)(n(2n+1) + 6(n+1))}{6} \\
                &= \frac{(n+1)(2n^2 + n + 6n + 6)}{6} \\
                &= \frac{(n+1)(2n^2 + 7n + 6)}{6} \\
                &= \frac{(n+1)(n+2)(2n+3)}{6} \\
                &= \frac{(n+1)(n+2)(2(n+1)+1)}{6}
            \end{align*}
            Por lo tanto, se cumple $\varphi(n+1)$; finalizando el paso inductivo.
        \end{enumerate}
        Debido al primer principio de inducción, se concluye que $\forall n \in \mathbb{N} \big( \varphi(n) \big)$.
    \end{proof}

    \begin{ejercicio}{3}
        Sean $A$  y $B$ conjuntos, con $B \subseteq A$. Prueba que, si $B$ es finito y $A$ infinito, entonces $A \setminus B$ es infinito.
    \end{ejercicio}
    \begin{proof}
        Por contradicción, supongamos que $A$ es infinito, $B \subseteq A$ finito y que $A \setminus B$ es finito. Dado que $B \subseteq A$, entonces $A=(A \setminus B) \cup B$, de donde, $A$ es unión de dos conjuntos finitos (y ajenos), con ello, $A$ es finito. Esto contradice la hipótesis de que $A$ es infinito; por lo tanto ocurre la negación de ``$A$ es infinito, $B \subseteq A$ finito y $A \setminus B$ es finito''; equivalentemente, ``si $B$ es finito y $A$ infinito, entonces $A \setminus B$ es infinito''.
    \end{proof}

    \begin{ejercicio}{3}
        Sean $x,y \in \mathbb{R}$ y $a_{900},a_{899}, \dots ,a_1,a_0$ son los coeficientes (en orden) del polinomio $(x+y)^{900}$; es decir $ (x+y)^{900} = a_{900} x^{900} + a_{899} x^{899}y + \cdots + a_1 xy^{899} + a_0 y^{900}. $ ?`Cuál de los siguientes números es mayor, $a_{100}$ o $a_{798}$? Demuestra todas tus afirmaciones.
    \end{ejercicio}
    \begin{solucion}
        Se afirma que $a_{100}<a_{798}$. Por el Teorema del Binomio de Newton, se tiene que:
        \[ a_{100} = \binom{900}{100} \quad \text{y} \quad a_{798} = \binom{900}{798} \]
        Y por propiedades de los coeficientes binomiales: $\quad a_{798} = \binom{900}{798} = \binom{900}{900-798} = \binom{900}{102}$. Ahora, notemos que:
        \begin{align*}
            \frac{a_{100}}{a_{798}} &= \frac{\binom{900}{100}}{\binom{900}{102}} \\
            &= \frac{900!}{100!(900-100)!} \cdot \frac{102!(900-102)!}{900!} \\
            &= \frac{102!}{100!} \cdot \frac{(900-102)!}{(900-100)!} \\
            &= \frac{102 \cdot 101}{(900-100)(900-102)} \\
            &= \frac{102 \cdot 101}{800 \cdot 798} \\
            &< 1
        \end{align*}
        Por lo tanto, $a_{100} < a_{798}$.
    \end{solucion}

    \begin{ejercicio}{+1}
        Sea $A \subseteq \mathbb{N}$ y supongamos que $\forall x \big( x \in A \to x+1 \in A \big)$. Prueba que si $m \in \mathbb{N} \cap A$, entonces $\{ n \in \mathbb{N} \mid n \geq m \} \subseteq A$.
    \end{ejercicio}
    \begin{proof}
        Sea $\varphi(n)$ la propiedad: $n \in A$. Observemos que:
        \begin{align*}
            \{ n \in \mathbb{N} \mid n \geq m \} \subseteq A & \Leftrightarrow \forall n \in \mathbb{N} \big( n \geq m \to n \in A \big) \\
            & \Leftrightarrow \forall n \in \mathbb{N} \big( n \geq m \to n \in A \big) \\
            & \Leftrightarrow \forall n \geq m \big( n \in A \big)
        \end{align*}
        
        Por tanto, para demostrar lo que nos ataña, basta probar por inducción (a partir de un punto) que $\forall n \geq m \big( \varphi(n) \big)$.
        \begin{enumerate}[\hspace{1cm}]
            \item \underline{Base.} Se cumple $\varphi(m)$; efectivamente, esto es simplemente porque ``$m \geq m$'' es verdadera y ``$m \in A$'' también (pues por hipótesis general, $m \in \mathbb{N} \cap A$).
            \item \underline{Paso inductivo.} Sea $n \geq m$ y supongamos que $\varphi(n)$ (H.I.), esto es, $n \in A$. Veamos que $\varphi(n+1)$, es decir, que $n+1 \in A$. Esto es inmediato de la hipótesis general, recordemos que $\forall x \big( x \in A \to x+1 \in A \big)$. Lo cual prueba el paso inductivo.
        \end{enumerate}
        Debido al primer principio de inducción, se concluye que $\forall n \geq m \big( \varphi(n) \big)$, es decir, se cumple la contención $\{ n \in \mathbb{N} \mid n \geq m \} \subseteq A$.
    \end{proof}

\end{document}