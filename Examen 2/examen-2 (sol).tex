\documentclass[letterpaper,DIV=14,headsepline,12pt]{scrartcl}
\usepackage[spanish,mexico,shorthands=off,es-lcroman]{babel}
\usepackage{stix2}
\pagenumbering{gobble}
%\setlength{\parindent}{0cm}
\usepackage{mathtools}
\usepackage{amsthm}
\usepackage{tikz-cd}
\usepackage{comment}
\usepackage{lipsum}
\usepackage{ifthen}
\usepackage{truthtable} %para las tablas de verdad
\usepackage[table]{xcolor}

\usepackage{multicol}
\usepackage{paracol}
\usepackage[shortlabels]{enumitem}
\setenumerate[1]{label=\MakeLowercase{\roman*}), ref=\roman*}
\setenumerate[2]{label=\MakeLowercase{\arabic*}), ref=\alph*}

% Para escribir el "tal que" de los conjuntos
\providecommand\st{\;|\;}
\providecommand\tq{\;|\;}

%Para el uso de \Set y \Set*
\providecommand\given{}
\newcommand\SetSymbol[1][]{\nonscript\:#1\vert\allowbreak\nonscript\:\mathopen{}}
\DeclarePairedDelimiterX\Set[1]\{\}{\renewcommand\given{\SetSymbol[\delimsize]}#1}
\DeclarePairedDelimiterX\Par[1](){#1}

\newcounter{Ejer}
\newcounter{Pts}
\setcounter{Ejer}{1}
\setcounter{Pts}{1}
\newcommand{\pts}{}
\newenvironment{ejercicio}[1]{\noindent
    \ifthenelse{\equal{#1}{1} \OR \equal{#1}{+1}}{\renewcommand{\pts}{\textbf{(#1 pt)}}}{\renewcommand{\pts}{\textbf{(#1 pts)}}}\textbf{Ej. \theEjer} \pts\stepcounter{Ejer}}{\vspace{.3cm}}

%Comandos que utilizamos
\newcommand{\id}{\mathrm{id}}
\newcommand{\op}{{}^{\mathrm{op}}}
\newcommand{\set}[1]{\{#1\}}
\renewcommand{\emptyset}{\varnothing}
\DeclareMathOperator{\ima}{ima}
\DeclareMathOperator{\dom}{dom}
\newcommand{\quot}[2]{{\raisebox{.2em}{$#1$}\left/\raisebox{-.2em}{$#2$}\right.}}

%Colors
\definecolor{azul}{RGB}{0, 60, 113}
\definecolor{dorado}{RGB}{196, 151, 57}
\definecolor{morado}{RGB}{101, 43, 145}

%Entorno de Demostración y Solución
\renewcommand\qedsymbol{$\blacksquare$}
\makeatletter
    \renewenvironment{proof}[1][]{%
        \par\pushQED{\qed}%
        \normalfont\topsep6pt \partopsep0pt % espacio antes del entorno
        \trivlist
        \item[\hskip\labelsep
                \textbf{\textit{Demostración.}}% Cambiado aquí
        ]#1
        }{%
        \popQED\endtrivlist\@endpefalse
    }
\makeatother

\makeatletter
    \newenvironment{solucion}[1][]{%
        \par\pushQED{\hfill \lozenge}%
        \normalfont\topsep6pt \partopsep0pt % espacio antes del entorno
        \trivlist
        \item[\hskip\labelsep
                \textbf{\textit{Solución.}}% Cambiado aquí
        ]#1
        }{%
        \popQED\endtrivlist\@endpefalse
    }
\makeatother

\begin{document}

    \begin{center}
        {\fontsize{30}{60}\rmfamily \textbf{Segundo Examen (Solución)}} \\ \vspace{.2cm}
            Álgebra Superior 1, 2025-4
    \end{center}

    \begin{flushright}
        \footnotesize \hfill Profesor: Luis Jesús Trucio Cuevas.\\
        \hfill Ayudante: Hugo Víctor García Martínez.
    \end{flushright}

    \begin{ejercicio}{2.5}
        Demuestra que la relación \(R\subseteq A\times A\) es transitiva y simétrica si y
        sólo si \(R^{-1}\circ R=R\).
    \end{ejercicio}
    \begin{proof}
        ($\Rightarrow$) Supongamos que $R \subseteq A \times A$ es transitiva y simétrica, veamos$R^{-1}\circ R=R$.
        \begin{enumerate}[\hspace{.3cm}]
            \item ($\subseteq$) Sea $(x,y) \in R^{-1}\circ R$, entonces existe $z \in A$ de modo que $(x,z) \in R$ y $(z,y) \in R^{-1}$. Por definición de relación iniversa, $(y,z) \in R$ y por ser $R$ simétrica, $(z,x) \in R$. Luego $(y,z),(z,x) \in R$ y $R$ es transitiva, por lo que $(y,x) \in R$, pero como $R$ es simétrica, $(x,y) \in R$. Esto prueba que $R^{-1}\circ R \subseteq R$.
            
            ($\supseteq$) Sea $(a,b) \in R$, como $R$ es simetrica, $(b,a) \in R$ y por ello $(a,b) \in R^{-1}$. Por otro lado, como $(a,b),(b,a) \in R$ y $R$ es transitiva, entonces $(a,a) \in R$. Así, si $c=a$, entonces $(a,c) \in R$ y $(c,b) \in R^{-1}$; lo cual demuestra que $(a,b) \in R^{-1} \circ R$. Por lo tanto $R \subseteq R^{-1} \circ R$, y con ello:
            \[ R=R^{-1}\circ R \]
        \end{enumerate}

        ($\Leftrightarrow$) Supongamos que $R=R^{-1}\circ R$, veamos que $R$ es simiétrica y transitiva.
        \begin{enumerate}[\hspace{.3cm}]
            \item (Simetría) Sean $x,y \in A$ y supongamos que $(x,y) \in R$, entonces se sigue de la hipótesis que $(x,y) \in R^{-1}\circ R$ y así, existe $z \in A$ de modo que $(x,z) \in R$ y $(z,y) \in R^{-1}$. Por definición de relación inversa, $(y,z)\in R$ y $(z,x) \in R^{-1}$, de donde $(y,x) \in R^{-1} \circ R$, lo cual implica por hipótesis que $(y,x) \in R$. Por lo que$R$ es simétrica.
            
            \item (Transitividad) Sean $a,b,c \in A$ y supongamos que $(a,b) \in R$ y $(b,c) \in R$. Como $R$ es simétrica (probado en el párrafo de arriba), de lo último se obtiene que $(c,b) \in R$ y con ello $(b,c) \in R^{-1}$. Luego, $(a,b) \in R$ y $(b,c) \in R^{-1}$, así que $(a,c) \in R^{-1} \circ R$, de donde $(a,c) \in R$. Por lo tanto, $R$ es simétrica.
        \end{enumerate}
    \end{proof}


    \newpage
    \begin{ejercicio}{2.5}
        Sea \(f\colon A\to B\) una función. Demuestra que 
        \(A=\bigcup_{b\in B}f^{-1}[\Set{b}]\). 
    \end{ejercicio}

    \begin{ejercicio}{2.5}
        Sean \(f\colon A\to B\) una función y \(S\subseteq A\). Demuestra que si
        \(f\) es inyectiva, entonces \(f^{-1}[f[S]]=S\).
    \end{ejercicio}

    \begin{ejercicio}{2.5}
        Sean \(f\colon A\to B\) y \(g,h\colon B\to A\) funciones. Demuestra que
        si \(g\) es inversa izquierda de \(f\) y \(h\) es inversa derecha de
        \(f\), entonces \(g=h\).
    \end{ejercicio}

    \begin{ejercicio}{+1}
        \textit{Este ejercicio es opcional y sólo se tomará en cuenta si no hay 
        errores en la solución.}
            
        Sean $X$ un conjunto y $g\colon\emptyset \to X$. Pruebe que las siguientes condiciones son equivalentes:
        \begin{enumerate}
            \item $g$ es biyectiva.
            \item $g$ es sobreyectiva.
            \item $X=\emptyset$.
        \end{enumerate}
    \end{ejercicio}
    
\end{document}