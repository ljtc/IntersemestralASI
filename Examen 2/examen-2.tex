\documentclass[letterpaper,DIV=16,headsepline,11pt]{scrartcl}
\usepackage[spanish,mexico]{babel}
\pagenumbering{gobble}
%\setlength{\parindent}{0cm}

\usepackage{mathtools}
\usepackage{amsthm}
\usepackage{ifthen}
\usepackage{unicode-math}
\setmainfont{STIX Two Text}
\setmathfont{STIX Two Math}

\usepackage{tasks}
%Para repetir un bloque sin tener que repetir el código
%Falta mejorar, es una caja de LaTeX, por lo que no se puede romper
\ExplSyntaxOn
\box_new:N \l_repetir_caja
\NewDocumentEnvironment{Repeat}{m +b}{
  \box_clear:N \l_repetir_caja
  \vbox_set:Nn \l_repetir_caja {#2}
  \prg_replicate:nn {#1} {\box_use:N \l_repetir_caja }
}{}
\ExplSyntaxOff

% Para escribir el "tal que" de los conjuntos
\providecommand\st{\;|\;}

%Para el uso de \Set y \Set*
\providecommand\given{}
\newcommand\SetSymbol[1][]{\nonscript\:#1\vert\allowbreak\nonscript\:\mathopen{}}
\DeclarePairedDelimiterX\Set[1]\{\}{\renewcommand\given{\SetSymbol[\delimsize]}#1}
\DeclarePairedDelimiterX\Par[1](){#1}
%Ejercicios \newcommand{\pts}[1]{%
  %\ifthenelse{\equal{#1}{1}}{\hfill \textbf{(#1 pt)}}{\hfill\textbf{(#1 pts)}}
%}

\newcounter{Ejer}
\newcounter{Pts}
\setcounter{Ejer}{1}
\setcounter{Pts}{1}
\newcommand{\pts}{}
\newenvironment{ejercicio}[1]{\noindent
    \ifthenelse{\equal{#1}{1} \OR \equal{#1}{+1}}{\renewcommand{\pts}{\textbf{(#1
    pt)}}}{\renewcommand{\pts}{\textbf{(#1 pts)}}}\textbf{Ej. \theEjer}
    \pts\stepcounter{Ejer}}{\vspace{.3cm}}

%Comandos que utilizamos
\newcommand{\id}{\mathrm{id}}
\newcommand{\set}[1]{\{#1\}}
\renewcommand{\emptyset}{\varnothing}
\DeclareMathOperator{\ima}{ima} \DeclareMathOperator{\dom}{dom}

\begin{document}

\begin{Repeat}{2}
\begin{center}
  {\huge \textbf{Segundo Examen Parcial}} \\
  Álgebra Superior 1, 2025-4
\end{center}
%\begin{flushright}
%  \footnotesize \hfill Profesor: Luis Jesús Trucio Cuevas.\\
%  \hfill Ayudante: Hugo Víctor García Martínez.
%\end{flushright}

\noindent\textit{\textbf{Instrucciones.} Resuelve los siguientes ejercicios, se pueden utilizar libremente resultados vistos en clase, siempre y cuando, se indique claramente dónde y cuáles se utilizan.}\vspace{.4cm}

\begin{ejercicio}{2.5}
  Demuestra que la relación \(R\subseteq A\times A\) es transitiva y simétrica si y
  sólo si \(R^{-1}\circ R=R\).
\end{ejercicio}

\begin{ejercicio}{2.5}
  Sea \(f\colon A\to B\) una función. Demuestra que 
  \(A=\bigcup_{b\in B}f^{-1}[\Set{b}]\). 
\end{ejercicio}

\begin{ejercicio}{2.5}
  Sean \(f\colon A\to B\) una función y \(S\subseteq A\). Demuestra que si
  \(f\) es inyectiva, entonces \(f^{-1}[f[S]]=S\).
\end{ejercicio}

\begin{ejercicio}{2.5}
  Sean \(f\colon A\to B\) y \(g,h\colon B\to A\) funciones. Demuestra que
  si \(g\) es inversa izquierda de \(f\) y \(h\) es inversa derecha de
  \(f\), entonces \(g=h\).
\end{ejercicio}

\begin{ejercicio}{+1}
  \textit{Este ejercicio es opcional y sólo se tomará en cuenta si no hay 
  errores en la solución.}
    
  Sean $X$ un conjunto y $g\colon\emptyset \to X$. Pruebe que las siguientes condiciones son equivalentes:
  \begin{tasks}[label=\roman*),label-width=3ex](3)
    \task $g$ es biyectiva.
    \task $g$ es sobreyectiva.
    \task $X=\emptyset$.
  \end{tasks}
\end{ejercicio}

\vspace{0.8cm}
\end{Repeat}

\end{document}