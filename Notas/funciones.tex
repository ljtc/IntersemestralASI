\documentclass[letterpaper,DIV=14,headsepline,12pt]{scrartcl}
\usepackage[spanish,mexico,shorthands=off,es-lcroman]{babel}
\usepackage{stix2}
\pagenumbering{gobble}
%\setlength{\parindent}{0cm}
\usepackage{mathtools}
\usepackage{amsthm}
\usepackage{tikz-cd}
\usepackage{comment}
\usepackage{lipsum}
\usepackage{ifthen}
%\usepackage{truthtable} %para las tablas de verdad
\usepackage[x11names, table]{xcolor}
\usepackage[colorlinks=true, allcolors=VioletRed2]{hyperref}
\newcommand{\customRef}[2]{\hyperref[#1]{#2~\ref*{#1}}}


\usepackage{multicol}
\usepackage{paracol}
\usepackage[shortlabels]{enumitem}
\setenumerate[1]{label=\MakeLowercase{\roman*}), ref=\roman*}
\setenumerate[2]{label=\MakeLowercase{\arabic*}), ref=\alph*}

% Para escribir el "tal que" de los conjuntos
\providecommand\st{\;|\;}
\providecommand\tq{\;|\;}

%Para el uso de \Set y \Set*
\providecommand\given{}
\newcommand\SetSymbol[1][]{\nonscript\:#1\vert\allowbreak\nonscript\:\mathopen{}}
\DeclarePairedDelimiterX\Set[1]\{\}{\renewcommand\given{\SetSymbol[\delimsize]}#1}
\DeclarePairedDelimiterX\Par[1](){#1}

\newcounter{Ejer}
\newcounter{Pts}
\setcounter{Ejer}{1}
\setcounter{Pts}{1}
\newcommand{\pts}{}
\newenvironment{ejercicio}[1]{\noindent
    \ifthenelse{\equal{#1}{1} \OR \equal{#1}{+1}}{\renewcommand{\pts}{\textbf{(#1 pt)}}}{\renewcommand{\pts}{\textbf{(#1 pts)}}}\textbf{Ej. \theEjer} \pts\stepcounter{Ejer}}{\vspace{.3cm}}

%Comandos que utilizamos
\newcommand{\id}{\mathrm{id}}
\newcommand{\op}{{}^{\mathrm{op}}}
\newcommand{\set}[1]{\{#1\}}
\renewcommand{\emptyset}{\varnothing}
\DeclareMathOperator{\ima}{ima}
\DeclareMathOperator{\dom}{dom}
\newcommand{\quot}[2]{{\raisebox{.2em}{$#1$}\left/\raisebox{-.2em}{$#2$}\right.}}

%Colors
\definecolor{azul}{RGB}{0, 60, 113}
\definecolor{dorado}{RGB}{196, 151, 57}
\definecolor{morado}{RGB}{101, 43, 145}

%Cajas para teoremas y esas cosas
\usepackage{thmtools}
\usepackage[framemethod=TikZ]{mdframed}
\mdfsetup{innertopmargin=-2pt, innerbottommargin=5pt, roundcorner=5pt, linewidth=0}

\newmdtheoremenv[backgroundcolor=morado!8]{definicion}{Definición}
\newmdtheoremenv[backgroundcolor=azul!8]{recordatorio}[definicion]{Recordatorio}
\newmdtheoremenv[backgroundcolor=dorado!8]{teorema}[definicion]{Teorema}
\newmdtheoremenv[backgroundcolor=dorado!8]{observacion}[definicion]{Observación}

%Entorno de Demostración y Solución
\renewcommand\qedsymbol{$\blacksquare$}
\makeatletter
    \renewenvironment{proof}[1][]{%
        \par\pushQED{\qed}%
        \normalfont\topsep6pt \partopsep0pt % espacio antes del entorno
        \trivlist
        \item[\hskip\labelsep
                \textbf{\textit{Demostración.}}% Cambiado aquí
        ]#1
        }{%
        \popQED\endtrivlist\@endpefalse
    }
\makeatother

\makeatletter
    \newenvironment{solu}[1][]{%
        \par\pushQED{\hfill \lozenge}%
        \normalfont\topsep6pt \partopsep0pt % espacio antes del entorno
        \trivlist
        \item[\hskip\labelsep
                \textbf{\textit{Solución.}}% Cambiado aquí
        ]#1
        }{%
        \popQED\endtrivlist\@endpefalse
    }
\makeatother

\begin{document}

    \begin{center}
        {\fontsize{30}{60}\rmfamily \textbf{``Definicionario'' de funciones.}} \\ \vspace{.2cm}
        Álgebra Superior 1, 2025-4
    \end{center}
    \begin{flushright}
        \footnotesize \hfill Profesor: Luis Jesús Trucio Cuevas.\\
        \hfill Ayudante: Hugo Víctor García Martínez.
    \end{flushright}

    \begin{definicion}
        Sean $A$ y $B$ conjuntos. Una \textbf{función} de $A$ a $B$ es una relación $f \subseteq A \times B$ tal que:
        \begin{enumerate}
            \item $\dom(f)=A$.
            \item Para todo $a \in A$, existe un único $b \in B$ tal que $a \mathrel{f} b$, \textit{como tal $b$ es único, se le puede dar una notación especial: $f(a)$}.
        \end{enumerate}
        Cuando $f$ sea función de $A$ en $B$, se escribirá $f: A \to B$. Además, se conviene que:
        \begin{enumerate} \setcounter{enumi}{3}
            \item \textbf{UN codominio} para $f$ es cualquier conjunto $Y$ de modo que $\ima(f) \subseteq Y$.
        \end{enumerate}
    \end{definicion}

    En virtud del punto (ii) de la definición anterioro, debe ser claro que si $f:A \to B$:
    \[ f=\{(x,f(x)) \tq x \in A \} \]

    Pese a la ``definición formal'' de función, en la práctica, pensaremos a la función como un objeto que cuenta con:
    \begin{enumerate}
        \item Un nombre,
        \item Un dominio,
        \item Una codominio (esto es, cualquier conjunto $B$ tal que $\ima(f)\subseteq B$), y
        \item Una ``regla de correspondencia'' (la instrucción que dicta cómo actúa la función en cada elemento del dominio).
    \end{enumerate}

    \begin{observacion}
        Una función puede tener varios codominios, por ejemplo, $f:\mathbb{N} \to \mathbb{Z}$ definida por $f(n)=4n+5$ también puede ser pensada como función con codominio $\mathbb{Q}$, $\mathbb{N}$, etcétera; pues, todos estos conjuntos contienen a la imagen de $f$: $\{ 4n+5 \tq n \in \mathbb{N} \}$. Entonces, es válido pensar a $f$ como:
        \[ f:\mathbb{N} \to \mathbb{Z} \text{, } f:\mathbb{N} \to \mathbb{Q} \text{, } f:\mathbb{N} \to \mathbb{N} \text{, o } f:\mathbb{N} \to \mathbb{R} \text{ etc..} \]

        Sin embargo el dominio de cualquier función sí es único (al igual que su imagen), esto es un hecho inmediato a las definiciones de estos conjuntos.
    \end{observacion}

    En la práctica, rara vez verificaremos la igualdad de funciones mediante doble contención o métodos similares, recordemos que se tiene el siguiente criterio:

    \begin{recordatorio}
        Si $f:A \to B$ y $g:X \to Y$ son funciones, entonces $f=g$ si y solamente si se cumplen las siguientes dos condiciones:
        \begin{enumerate}
            \item $\dom(f)=\dom(g)$; es decir, $A=X$; y
            \item Para cada $x \in \dom(f)$, se cumple que $f(x)=g(x)$.
        \end{enumerate}
    \end{recordatorio}

    \begin{definicion}
        Sean $A$, $B$ conjuntos y $f: A \to B$ una función. Se dice que:
        \begin{enumerate}
            \item $f$ es \textbf{inyectiva} si y sólo si $\forall x,y \in A \big( f(x)=f(y) \to x=y \big)$.
            \item $f$ es \textbf{sobreyectiva} (en $B$) si y sólo si $\forall b \in B \, \exists a \in \big( b=f(a) \big)$. \textit{El término ``sobreyectiva'' (a secas) se refiere a la sobreyectividad en el codominio indicado al momento de dar la función.}
            \item $f$ es \textbf{biyectiva} si y sólo si es inyectiva y sobreyectiva.
        \end{enumerate}
    \end{definicion}

    \begin{definicion}
        Sean $f: A \to B$ y $g: B \to C$ funciones. Se define la función $g \circ f:A \to C$, para cada $a \in A$, por medio de:
        \[ \big( g \circ f \big)(a) = g\big( f(a) \big) \]
    \end{definicion}

    \begin{observacion}
        Para que dos funcines $f$ y $g$ se puedan componer, aplicando (por ejemplo) primero $f$ y luego $g$, tal composición debe tener sentido; y para ello, es requisito que:
        \[ \ima(f) \subseteq \dom(g) \]

        Es decir, basta que \textit{algún} codominio de $f$ (recordemos que hay muchos codominios), esté contenido en el dominio de $g$.
    \end{observacion}

    
    

\end{document}