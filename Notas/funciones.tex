\documentclass[letterpaper,DIV=14,headsepline,12pt]{scrartcl}
\usepackage[spanish,mexico,shorthands=off,es-lcroman]{babel}
\usepackage{stix2}
%\setlength{\parindent}{0cm}
\usepackage{mathtools}
\usepackage{amsthm}
\usepackage{tikz-cd}
\usepackage{comment}
\usepackage{lipsum}
\usepackage{ifthen}
%\usepackage{truthtable} %para las tablas de verdad
\usepackage[x11names, table]{xcolor}
\usepackage[colorlinks=true, allcolors=VioletRed2]{hyperref}
\newcommand{\customRef}[2]{\hyperref[#1]{#2~\ref*{#1}}}

\usepackage{scrlayer-scrpage}
    \clearpairofpagestyles
    \ihead{\footnotesize \textit{Álgebra Superior I}}
    \ohead{\footnotesize \textit{Intersemestral 2025-4}}
    \cfoot{\normalfont\thepage}
    \addtokomafont{title}{\bfseries \rmfamily}
    \setlength{\headsep}{5pt}
    \newpairofpagestyles{beginstyle}{
        \clearpairofpagestyles
        \KOMAoptions{headsepline=false}
        \cfoot{\footnotesize \pagemark}
        \cfoot{\normalfont\thepage}
    }

\addtokomafont{section}{\raggedleft \rmfamily}
\addtokomafont{subsection}{\raggedleft\rmfamily}
\setlength{\headsep}{5pt}
\setlength{\footskip}{20pt}
\setlength{\textheight}{450pt}

\usepackage{multicol}
\usepackage{paracol}
\usepackage[shortlabels]{enumitem}
\setenumerate[1]{label=\MakeLowercase{\roman*}), ref=\roman*}
\setenumerate[2]{label=\MakeLowercase{\arabic*}), ref=\alph*}

% Para escribir el "tal que" de los conjuntos
\providecommand\st{\;|\;}
\providecommand\tq{\;|\;}

%Para el uso de \Set y \Set*
\providecommand\given{}
\newcommand\SetSymbol[1][]{\nonscript\:#1\vert\allowbreak\nonscript\:\mathopen{}}
\DeclarePairedDelimiterX\Set[1]\{\}{\renewcommand\given{\SetSymbol[\delimsize]}#1}
\DeclarePairedDelimiterX\Par[1](){#1}

\newcounter{Ejer}
\newcounter{Pts}
\setcounter{Ejer}{1}
\setcounter{Pts}{1}
\newcommand{\pts}{}
\newenvironment{ejercicio}[1]{\noindent
    \ifthenelse{\equal{#1}{1} \OR \equal{#1}{+1}}{\renewcommand{\pts}{\textbf{(#1 pt)}}}{\renewcommand{\pts}{\textbf{(#1 pts)}}}\textbf{Ej. \theEjer} \pts\stepcounter{Ejer}}{\vspace{.3cm}}

%Comandos que utilizamos
\newcommand{\id}{\mathrm{id}}
\newcommand{\op}{{}^{\mathrm{op}}}
\newcommand{\set}[1]{\{#1\}}
\renewcommand{\emptyset}{\varnothing}
\DeclareMathOperator{\ima}{ima}
\DeclareMathOperator{\dom}{dom}
\newcommand{\quot}[2]{{\raisebox{.2em}{$#1$}\left/\raisebox{-.2em}{$#2$}\right.}}

%Colors
\definecolor{azul}{RGB}{0, 60, 113}
\definecolor{dorado}{RGB}{196, 151, 57}
\definecolor{morado}{RGB}{101, 43, 145}

%Cajas para teoremas y esas cosas
\usepackage{thmtools}
\usepackage[framemethod=TikZ]{mdframed}
\mdfsetup{innertopmargin=-2pt, innerbottommargin=5pt, roundcorner=5pt, linewidth=0}

\newmdtheoremenv[backgroundcolor=morado!10]{definicion}{Definición}
\newmdtheoremenv[backgroundcolor=RoyalBlue2!10]{recordatorio}[definicion]{Recordatorio}
\newmdtheoremenv[backgroundcolor=teal!10]{teorema}[definicion]{Teorema}
\newmdtheoremenv[backgroundcolor=dorado!10]{observacion}[definicion]{Observación}

%Entorno de Demostración y Solución
\renewcommand\qedsymbol{$\blacksquare$}
\makeatletter
    \renewenvironment{proof}[1][]{%
        \par\pushQED{\qed}%
        \normalfont\topsep6pt \partopsep0pt % espacio antes del entorno
        \trivlist
        \item[\hskip\labelsep
                \textbf{\textit{Demostración.}}% Cambiado aquí
        ]#1
        }{%
        \popQED\endtrivlist\@endpefalse
    }
\makeatother

\makeatletter
    \newenvironment{solu}[1][]{%
        \par\pushQED{\hfill \lozenge}%
        \normalfont\topsep6pt \partopsep0pt % espacio antes del entorno
        \trivlist
        \item[\hskip\labelsep
                \textbf{\textit{Solución.}}% Cambiado aquí
        ]#1
        }{%
        \popQED\endtrivlist\@endpefalse
    }
\makeatother

\begin{document}

    \pagestyle{scrheadings}
    \thispagestyle{beginstyle}
    \begin{center}
        {\fontsize{30}{60}\rmfamily \textbf{``Definicionario'' de funciones.}}
    \end{center}
    \begin{flushright}
        \footnotesize \hfill Profesor: Luis Jesús Trucio Cuevas.\\
        \hfill Ayudante: Hugo Víctor García Martínez.
    \end{flushright}

    \tableofcontents

    \newpage
    \section{Definiciones básicas}

    \begin{definicion}
        Sean $A$ y $B$ conjuntos. Una \textbf{función} de $A$ a $B$ es una relación $f \subseteq A \times B$ tal que:
        \begin{enumerate}
            \item $\dom(f)=A$.
            \item Para todo $a \in A$, existe un único $b \in B$ tal que $a \mathrel{f} b$, \textit{como tal $b$ es único, se le puede dar una notación especial: $f(a)$}.
        \end{enumerate}
        Cuando $f$ sea función de $A$ en $B$, se escribirá $f: A \to B$. Además, se conviene que:
        \begin{enumerate} \setcounter{enumi}{3}
            \item \textbf{UN codominio} para $f$ es cualquier conjunto $Y$ de modo que $\ima(f) \subseteq Y$.
        \end{enumerate}
    \end{definicion}

    En virtud del punto (ii) de la definición anterioro, debe ser claro que si $f:A \to B$, entonces $f=\{(x,f(x)) \tq x \in A \}$.

    Pese a la ``definición formal'' de función, en la práctica, pensaremos a la función como un objeto que cuenta con un nombre, un dominio, un codominio (esto es, cualquier conjunto $B$ tal que $\ima(f)\subseteq B$), y una ``regla de correspondencia'' (la instrucción que dicta cómo actúa la función en cada elemento del dominio).

    \begin{observacion}
        Una función puede tener varios codominios, por ejemplo, $f:\mathbb{N} \to \mathbb{Z}$ definida por $f(n)=4n+5$ también puede ser pensada como función con codominio $\mathbb{Q}$, $\mathbb{N}$, etcétera; pues, todos estos conjuntos contienen a la imagen de $f$: $\{ 4n+5 \tq n \in \mathbb{N} \}$. Entonces, es válido pensar a $f$ como:
        \[ f:\mathbb{N} \to \mathbb{Z} \text{, } f:\mathbb{N} \to \mathbb{Q} \text{, } f:\mathbb{N} \to \mathbb{N} \text{, o } f:\mathbb{N} \to \mathbb{R} \text{, etc...} \]

        Sin embargo el dominio de cualquier función sí es único (al igual que su imagen).
    \end{observacion}

    Rara vez verificaremos la igualdad de funciones mediante doble contención o métodos similares, se tiene el siguiente criterio:

    \begin{recordatorio}
        Si $f:A \to B$ y $g:X \to Y$ son funciones, entonces $f=g$ si y solamente si se cumplen las siguientes dos condiciones:
        \begin{enumerate}
            \item $\dom(f)=\dom(g)$; es decir, $A=X$; y
            \item Para cada $x \in \dom(f)$, se cumple que $f(x)=g(x)$.
        \end{enumerate}
    \end{recordatorio}

    \begin{definicion}
        Sean $A$, $B$ conjuntos y $f: A \to B$ una función. Se dice que:
        \begin{enumerate}
            \item $f$ es \textbf{inyectiva} si y sólo si $\forall x,y \in A \big( f(x)=f(y) \to x=y \big)$.
            \item $f$ es \textbf{sobreyectiva} (en $B$) si y sólo si $\forall b \in B \, \exists a \in \big( b=f(a) \big)$. \textit{El término ``sobreyectiva'' (a secas) se refiere a la sobreyectividad en el codominio indicado al momento de dar la función.}
            \item $f$ es \textbf{biyectiva} si y sólo si es inyectiva y sobreyectiva.
        \end{enumerate}
    \end{definicion}

    

    \section{Funciones y subconjuntos}

    Cuando se tenga una función $f:A \to B$, evidentemente, se le puede aplicar $f$ a cada punto (elemento) $a$ de su dominio, $A$. Es decir, la función toma puntos de $A$ y devuelve puntos de $B$.
    
    Una idea interesante es la siguiente: ¿se le puede ``aplicar'' $f$ a un subconjunto $X$ de su dominio A?. La respuesta corta es \textit{no}, pues la función \textit{sólo se aplica a puntos de su dominio}, no a subconjuntos del mismo. Sin embargo, se puede recuperar parte de esta idea:

    \begin{definicion}
        Sean $f:A \to B$ una función (así redactado, es claro que $A$ y $B$ son conjuntos arbitrarios) y $X \subseteq A$. Se define la \textbf{imagen directa de $X$} (\textbf{bajo $f$}) como el conjunto:
        \[ f[X]:=\{ f(x) \in B \tq x \in X \} \]
        Es decir, $ f[X]:=\Set*{ l \in B \given \exists x \in X \big( l=f(x) \big)} $.

        \textit{Estos dos conjuntos son exactamente iguales, pero la notación es distinta, se puede tomar la que mejor se entienda (pues, de nuevo, son exactamente lo mismo).}
    \end{definicion}

    La siguiente observación es fundamental para escribir correctamente las demostraciones que involucren imágenes directas en ellas.

    \begin{observacion}
        Tomar elementos en cualquier imagen directa $f[\bigstar]$ se traduce a lo siguiente:
        \[ l \in f[\bigstar] \iff \exists x \in \bigstar \big( l=f(x) \big) \]

        Además, siempre que se tenga un elemento $y \in A$ se cumplirá:
        \[ y \in \bigstar \Rightarrow f(y) \in f[\bigstar] \]
        equivalentemente (usando contrapuesta): $f(y) \notin f[\bigstar] \Rightarrow y \notin \bigstar$.
    \end{observacion}

    En clase se demostró lo siguiente:
    \begin{teorema}
        Sean $f:A \to B$ una función y $X, Y \subseteq A$ arbitrario. Entonces:
        \begin{enumerate}
            \item $f[\emptyset]=\emptyset$.
            \item $f[A]=f[\dom(f)]=\ima(f)$.
            \item Si $X \subseteq Y$, entonces $f[X] \subseteq f[Y]$.
            \item $f[X \cup Y] = f[X] \cup f[Y]$.
            \item $f[X \cap Y] \subseteq f[X] \cap f[Y]$, pero \textbf{no siempre} $f[X] \cap f[Y] \subseteq f[X \cap Y]$.
            \item $f[X] \setminus f[Y] \subseteq f[X \setminus Y]$, pero \textbf{no siempre} $f[X \setminus Y] \subseteq f[X] \setminus f[Y]$.
        \end{enumerate}
    \end{teorema}

    Una idea ``dual'' a la anterior es la imagen inversa, cabe aclarar que, pese a lo sugerente que es la siguiente notación en ningún momento se está haciendo referencia a la función inversa de $f$ (y tampoco, a la relación inversa, pensando a $f$ como relación).

    \begin{definicion}
        Sean $f:A \to B$ una función y $Y \subseteq B$. Se define la \textbf{imagen inversa de $Y$} (\textbf{bajo $f$}) como el conjunto:
        \[ f^{-1}[Y]:=\{ x \in X \tq f(x) \in Y \} \]
    \end{definicion}

    



    \newpage
    \section{Composición de funciones}
    \begin{definicion}
        Sean $f: A \to B$ y $g: B \to C$ funciones. Se define la función $g \circ f:A \to C$, para cada $a \in A$, por medio de:
        \[ \big( g \circ f \big)(a) = g\big( f(a) \big) \]
    \end{definicion}

    \begin{observacion}
        Para que dos funcines $f$ y $g$ se puedan componer, aplicando (por ejemplo) primero $f$ y luego $g$, tal composición debe tener sentido; y para ello, es requisito que:
        \[ \ima(f) \subseteq \dom(g) \]

        Es decir, basta que \textit{algún} codominio de $f$ (recordemos que hay muchos codominios), esté contenido en el dominio de $g$.
    \end{observacion}
    
    

\end{document}