\documentclass[letterpaper,DIV=12,headsepline,12pt]{scrartcl}

%Header and footer
\usepackage{scrlayer-scrpage}
    \clearpairofpagestyles
    \ihead{\footnotesize \textit{Álgebra Superior I}}
    \ohead{\footnotesize \textit{Intersemestral 2025-4}}
    \cfoot{\normalfont\thepage}
    \addtokomafont{title}{\bfseries \rmfamily}
    \setlength{\headsep}{5pt}
    \newpairofpagestyles{beginstyle}{
        \clearpairofpagestyles
        \KOMAoptions{headsepline=false}
        \cfoot{\footnotesize \pagemark}
        \cfoot{\normalfont\thepage}
    }

\usepackage{iftex}
\usepackage[spanish,mexico,es-noindentfirst]{babel}
\usepackage{mathtools}
\usepackage{tasks}
\settasks{style=itemize}
\ifluatex
  \usepackage{unicode-math}
  \setmainfont{Stix Two Text}
  \setmathfont{Stix Two Math}
\else
  \usepackage[T1]{fontenc}
  \usepackage{stix2}
\fi

\begin{document}

\thispagestyle{beginstyle}
\begin{center}
  {\fontsize{30}{60}\rmfamily \textbf{Equivalencias e implicaciones lógicas}}
  \\ \vspace{.2cm}
  Álgebra Superior 1, 2025-4
\end{center}
\begin{flushright}
  \footnotesize \hfill Profesor: Luis Jesús Trucio Cuevas.\\
  \hfill Ayudante: Hugo Víctor García Martínez.
\end{flushright}

\noindent
Estas no son notas como tal, la intensión es tener a la mano algunas
equivalencias e implicaciones lógicas.

\section{Equivalencias}
\subsection{Estructurales}
La conjunción y la disyunción son conmutativas:
\begin{tasks}(2)
  \task \(\alpha\land\beta\equiv\beta\land\alpha\)
  \task \(\alpha\lor\beta\equiv\beta\lor\alpha\).
\end{tasks}
%
La conjunción y la disyunción son asociativas:
\begin{tasks}(2)
  \task \(\alpha\land(\beta\land\gamma)\equiv(\alpha\land\beta)\land\gamma\)
  \task \(\alpha\lor(\beta\lor\gamma)\equiv(\alpha\lor\beta)\lor\gamma\).
\end{tasks}
%
La conjunción y la disyunción son idempotentes:
\begin{tasks}(2)
  \task \(\alpha\land\alpha\equiv\alpha\)
  \task \(\alpha\lor\alpha\equiv\alpha\).
\end{tasks}
%
La conjunción y la disyunción son absorbentes:
\begin{tasks}(2)
  \task \(\alpha\land(\alpha\lor\beta)\equiv\alpha\)
  \task \(\alpha\lor(\alpha\land\beta)\equiv\alpha\).
\end{tasks}
%
La conjunción y la disyunción son distributivas:
\begin{tasks}(2)
  \task \(\alpha\land(\beta\lor\gamma)\equiv(\alpha\land\beta)\lor(\alpha\land\gamma)\)
  \task \(\alpha\lor(\beta\land\gamma)\equiv(\alpha\lor\beta)\land(\alpha\lor\gamma)\).
\end{tasks}
%
Denotemos con \(\top\) a una tautología y con \(\bot\) a una contradicción. Así,
\(\top\) es neutro para la conjunción y \(\bot\) es neutro para la disyunción:
\begin{tasks}(2)
  \task \(\alpha\land\top\equiv\alpha\)
  \task \(\alpha\lor\bot\equiv\alpha\).
\end{tasks}

\subsection{Lógica clásica}
Leyes de De Morgan:
\begin{tasks}(2)
  \task \(\neg(\alpha\land\beta)\equiv\neg\alpha\lor\neg\beta\)
  \task \(\neg(\alpha\lor\beta)\equiv\neg\alpha\land\neg\beta\).
\end{tasks}
%
Doble negación:
\begin{tasks}
  \task \(\neg\neg\alpha\equiv\alpha\).
\end{tasks}
%
Complementos:
\begin{tasks}(2)
  \task (tercero excluido) \(\top\equiv\alpha\lor\neg\alpha\)
  \task \(\bot\equiv\alpha\land\neg\alpha\).
\end{tasks}

\subsection{Conjuntos mínimos de conectivos}
Con las siguientes equivalencias es posible expresar cualquier proposición
lógica con sólo conectivos \(\lor\) y \(\neg\) (nota que para definir \(\bot\)
es necesario usar \(\neg\)):
\begin{tasks}(3)
  \task \(\alpha\to\beta\equiv\neg\alpha\lor\beta\)
  \task \(\alpha\land\beta\equiv\neg(\neg\alpha\lor\neg\beta)\)
  \task* \(\alpha\leftrightarrow\beta\equiv(\alpha\to\beta)\land(\beta\to\alpha)\).
\end{tasks}
¿Puedes escribir las equivalencias necesarias para escribir toda proposición con
sólo \(\land\) y \(\neg\) (o con sólo \(\lor\) y \(\neg\))?

\subsection{Métodos de demostración}
\begin{tasks}
  \task Contrapuesta: \(\alpha\to\beta\equiv\neg\beta\to\neg\alpha\).
  \task Contradicción: \(\alpha\to\beta\equiv(\alpha\land\neg\beta)\to\bot\).
  \task Dem de una negación: \(\neg\alpha\equiv\alpha\to\bot\).
  \task Dem de una conjunción: 
  \(\alpha\to(\beta\land\gamma)\equiv(\alpha\to\beta)\land(\alpha\to\gamma)\).
  \task Dem de una disyunción:
  \(\alpha\to(\beta\lor\gamma)\equiv(\alpha\land\neg\beta\to\gamma)\).
  \task Dem de un condicional:
  \((\alpha\to(\beta\to\gamma))\equiv(\alpha\land\beta)\to\gamma\).
  \task Dem de una bicondicional:
  \((\alpha\leftrightarrow\beta)\equiv(\alpha\to\beta)\land(\beta\to\alpha)\).
  \task Dem por casos:
  \((\alpha\lor\beta)\to\gamma\equiv(\alpha\to\gamma)\land(\beta\to\gamma)\).
\end{tasks}
\end{document}