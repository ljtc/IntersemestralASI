\documentclass[letterpaper,DIV=12,headsepline,12pt]{scrartcl}

%Header and footer
\usepackage{scrlayer-scrpage}
    \clearpairofpagestyles
    \ihead{\footnotesize \textit{Álgebra Superior I}}
    \ohead{\footnotesize \textit{Intersemestral 2025-4}}
    \cfoot{\normalfont\thepage}
    \addtokomafont{title}{\bfseries \rmfamily}
    \setlength{\headsep}{5pt}
    \newpairofpagestyles{beginstyle}{
        \clearpairofpagestyles
        \KOMAoptions{headsepline=false}
        \cfoot{\footnotesize \pagemark}
        \cfoot{\normalfont\thepage}
    }

\usepackage[spanish,mexico,es-noindentfirst]{babel}
\usepackage{mathtools}
\usepackage{tasks}
\settasks{style=itemize}
\usepackage{unicode-math}
\setmainfont{Stix Two Text}
\setmathfont{Stix Two Math}
\usepackage{microtype}

\begin{document}
\thispagestyle{beginstyle}
\begin{center}
  {\fontsize{30}{60}\rmfamily \textbf{Cuantificadores y conectivos}}
  \\ \vspace{.2cm}
  Álgebra Superior 1, 2025-4
\end{center}
\begin{flushright}
  \footnotesize \hfill Profesor: Luis Jesús Trucio Cuevas.\\
  \hfill Ayudante: Hugo Víctor García Martínez.
\end{flushright}

\noindent
Este documento es sólo una lista de equivalencias e implicaciones que muestran
cómo se comportan los cuantificadores con los conectivos lógicos.

\section{Cuantificadores y negación}
Estas equivalencias son las más usuales, así que empezamos con ellas:
\begin{tasks}(2)
  \task \(\neg\exists x\varphi(x)\iff\forall x\neg\varphi(x)\)
  \task \(\neg\forall x\varphi(x)\iff\exists x\neg\varphi(x)\).
\end{tasks}

\section{Con existencial}\label{sec:excon}
En este caso diremos que
no se lleva bien con la conjunción y sí se lleva bien con la disyunción:
\begin{tasks}(2)
  \task \(\exists x(\varphi(x)\land\psi(x))\implies\exists x\varphi(x)\land\exists x\psi(x)\)
  \task \(\exists x(\varphi(x)\lor\psi(x))\iff\exists x\varphi(x)\lor\exists x\psi(x)\).
\end{tasks}
En la primera implicación no se da el regreso. Para dar un ejemplo de eso,
tomamos como contexto \(\mathbb{N}\). Consideramos \(\varphi(x)\) como \(x\) es
par y \(\psi(x)\) como \(x\) es impar. En este caso, 
\(\exists x(\varphi(x)\land\psi(x))\) es falso ya que no hay números que
sean pares e impares al mismo tiempo, pero
\(\exists x\varphi(x)\land\exists x\psi(x)\) es cierto ya que hay números pares
e impares.

\section{Con universal}\label{sec:unicon}
En este caso diremos que se lleva bien con la conjunción y no se lleva bien con
la disyunción:
\begin{tasks}(2)
  \task \(\forall x(\varphi(x)\land\psi(x))\iff\forall x\varphi(x)\land\forall x\psi(x)\)
  \task \(\forall x\varphi(x)\lor\forall x\psi(x)\implies\forall x(\varphi(x)\lor\psi(x))\).
\end{tasks}
De nuevo, para ver que la implicación no es una equivalencia hay que tomar el
ejemplo de los pares e impares en \(\mathbb{N}\). En este caso el antecedente
dice que todo número es par o impar, lo cual es cierto. Sin embargo, el
consecuente dice que todo número es par o todo número es impar, lo cual es
falso.

\section{Addendum}
En las secciones \ref{sec:excon} y \ref{sec:unicon} hemos visto que el
cuantificador existencial no se lleva bien con la conjunción y que el universal
no se lleva bien con la disyunción. Sin embargo, hay casos en los que sí se
llevan bien. Por ejemplo, si una fórmula no tiene nada que ver con la variable
de cuantificación. Sea \(\psi\) una fórmula que no depende de \(x\), entonces
\begin{tasks}(2)
  \task \(\exists x(\varphi(x)\land\psi)\iff(\exists x\varphi(x))\land\psi\)
  \task \(\forall x(\varphi(x)\lor\psi)\iff(\forall x\varphi(x))\lor\psi\).
\end{tasks}

\section{Universal e implicación}
Posiblemente no sea tan común ver esta implicación, la añadimos por completud:
\begin{tasks}
  \task \(\forall x(\varphi(x)\to\psi(x))\implies(\forall x\varphi(x))\to(\forall x\psi(x))\).
\end{tasks}
¿Puedes dar un ejemplo donde \((\forall x\varphi(x))\to(\forall x\psi(x))\) sea
verdadero y \(\forall x(\varphi(x)\to\psi(x))\) falso?
\textit{Sigerencia:} De nuevo considera el contexto \(\mathbb{N}\). Luego,
intenta que \((\forall x\varphi(x))\to(\forall x\psi(x))\) sea verdadero porque
el antecedente es falso.

\end{document}