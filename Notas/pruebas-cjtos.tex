\documentclass[letterpaper,DIV=12,headsepline,12pt]{scrartcl}
\usepackage[spanish,mexico,shorthands=off,es-lcroman]{babel}
\usepackage{stix2}

%Header and footer
\usepackage{scrlayer-scrpage}
    \clearpairofpagestyles
    \ihead{\footnotesize \textit{Álgebra Superior I}}
    \ohead{\footnotesize \textit{Intersemestral 2025-4}}
    \cfoot{\normalfont\thepage}
    \addtokomafont{title}{\bfseries \rmfamily}
    \setlength{\headsep}{5pt}
    \newpairofpagestyles{beginstyle}{
        \clearpairofpagestyles
        \KOMAoptions{headsepline=false}
        \cfoot{\footnotesize \pagemark}
        \cfoot{\normalfont\thepage}
    }

\usepackage{mathtools}
\usepackage{amsthm}
\usepackage{tikz-cd}
\usepackage{comment}
\usepackage{lipsum}
\usepackage{ifthen}
\usepackage{truthtable} %para las tablas de verdad
\usepackage{xcolor}

\usepackage{multicol}
\usepackage[shortlabels]{enumitem}
\setenumerate[1]{label=\MakeLowercase{\roman*}), ref=\roman*}
\setenumerate[2]{label=\MakeLowercase{\arabic*}), ref=\alph*}

% Para escribir el "tal que" de los conjuntos
\providecommand\st{\;|\;}
\providecommand\tq{\;|\;}

%Para el uso de \Set y \Set*
\providecommand\given{}
\newcommand\SetSymbol[1][]{\nonscript\:#1\vert\allowbreak\nonscript\:\mathopen{}}
\DeclarePairedDelimiterX\Set[1]\{\}{\renewcommand\given{\SetSymbol[\delimsize]}#1}
\DeclarePairedDelimiterX\Par[1](){#1}

\newcounter{Ejer}
\newcounter{Pts}
\setcounter{Ejer}{1}
\setcounter{Pts}{1}
\newcommand{\pts}{}
\newenvironment{ejercicio}[1]{\noindent
    \ifthenelse{\equal{#1}{1}}{\renewcommand{\pts}{\textbf{(#1 pt)}}}{\renewcommand{\pts}{\textbf{(#1 pts)}}}\textbf{Ej. \theEjer} \pts\stepcounter{Ejer}}{\vspace{.3cm}}

%Comandos que utilizamos
\newcommand{\id}{\mathrm{id}}
\newcommand{\op}{{}^{\mathrm{op}}}
\newcommand{\set}[1]{\{#1\}}
\renewcommand{\emptyset}{\varnothing}
\DeclareMathOperator{\ima}{ima}
\DeclareMathOperator{\dom}{dom}

%Cajas para teoremas y esas cosas
\usepackage{thmtools}
\usepackage[framemethod=TikZ]{mdframed}
\mdfsetup{innertopmargin=-2pt, innerbottommargin=5pt, roundcorner=5pt, linewidth=0}

\definecolor{azul}{RGB}{0, 60, 113}
\definecolor{dorado}{RGB}{196, 151, 57}
\definecolor{morado}{RGB}{101, 43, 145}

\newmdtheoremenv[backgroundcolor=azul!8]{definicion}{Definición}
\newmdtheoremenv[backgroundcolor=azul!8]{proposicion}[definicion]{Proposición}
\newmdtheoremenv[backgroundcolor=azul!8]{lema}[definicion]{Lema}
\newmdtheoremenv[backgroundcolor=azul!8]{teorema}[definicion]{Teorema}
\newmdtheoremenv[backgroundcolor=azul!8]{corolario}[definicion]{Corolario}
\newmdtheoremenv[backgroundcolor=azul!8]{observacion}[definicion]{Observación}

%Entorno de Demostración
\renewcommand\qedsymbol{$\blacksquare$}
\makeatletter
    \renewenvironment{proof}[1][]{%
        \par\pushQED{\qed}%
        \normalfont\topsep6pt \partopsep0pt % espacio antes del entorno
        \trivlist
        \item[\hskip\labelsep
                \textbf{\textit{Demostración.}}% Cambiado aquí
        ]#1
        }{%
        \popQED\endtrivlist\@endpefalse
    }
\makeatother

\begin{document}

    \thispagestyle{beginstyle}
    \begin{center}
        {\fontsize{30}{60}\rmfamily \textbf{Pruebas sencillas de igualdades entre conjuntos}} \\ \vspace{.2cm}
        Álgebra Superior 1, 2025-4
    \end{center}
    \begin{flushright}
        \footnotesize \hfill Profesor: Luis Jesús Trucio Cuevas.\\
        \hfill Ayudante: Hugo Víctor García Martínez.
    \end{flushright}

    Veamos que, utilizando equivalencias lógicas, se pueden demostrar igualdades básicas entre conjuntos. La técnica es siempre: Traducir (utilizando las definiciones de las entre de conjuntos) lo que significa que un elemento arbitrario, \(x\), sea elemento de cierto conjunto; después, utilizar equivalencias lógicas de modo que nos acerquemos al objetivo de la prueba (y demostrarlas, si es necesario), y finalmente, volver a traducir las proposiciones en notación de conjuntos; por ejemplo, ``\(x \in A \land x \in B\)'' se traduce a ``\(x \in A \cap B\)''.

    Por ejemplo, recordando que

    \begin{proposicion}
        Dados \(A\) y \(X\) conjuntos, se cumple \(X \setminus (X \setminus A) = A\).
    \end{proposicion}
    \begin{proof}
        \begin{align}
            x \in X \setminus (X \setminus A) & \leftrightarrow \Par*{x \in X \land (x \notin X \setminus A)}
        \end{align}
    \end{proof}



\end{document}