\documentclass[letterpaper,DIV=12,headsepline,12pt]{scrartcl}

%Header and footer
\usepackage{scrlayer-scrpage}
    \clearpairofpagestyles
    \ihead{\footnotesize \textit{Álgebra Superior I}}
    \ohead{\footnotesize \textit{Intersemestral 2025-4}}
    \cfoot{\normalfont\thepage}
    \addtokomafont{title}{\bfseries \rmfamily}
    \setlength{\headsep}{5pt}
    \newpairofpagestyles{beginstyle}{
        \clearpairofpagestyles
        \KOMAoptions{headsepline=false}
        \cfoot{\footnotesize \pagemark}
        \cfoot{\normalfont\thepage}
    }

\usepackage[spanish,mexico]{babel}
\usepackage{stix2}
\usepackage{mathtools}
\usepackage{tasks}
\settasks{style=itemize}

\begin{document}

\thispagestyle{beginstyle}
\begin{center}
  {\fontsize{30}{60}\rmfamily \textbf{Equivalencias e implicaciones lógicas}}
  \\ \vspace{.2cm}
  Álgebra Superior 1, 2025-4
\end{center}
\begin{flushright}
  \footnotesize \hfill Profesor: Luis Jesús Trucio Cuevas.\\
  \hfill Ayudante: Hugo Víctor García Martínez.
\end{flushright}

Estas no son notas como tal, la intensión es tener a la mano algunas
equivalencias e implicaciones lógicas.

\section{Equivalencias}
La conjunción y la disyunción son conmutativas:
\begin{tasks}(2)
  \task \(\alpha\land\beta\equiv\beta\land\alpha\)
  \task \(\alpha\lor\beta\equiv\beta\lor\alpha\)
\end{tasks}

\end{document}