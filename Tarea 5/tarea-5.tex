\documentclass[letterpaper,DIV=14,headsepline,12pt]{scrartcl}
\usepackage[spanish,mexico,shorthands=off,es-lcroman]{babel}
\usepackage{stix2}
\pagenumbering{gobble}
%\setlength{\parindent}{0cm}

\usepackage{mathtools}
\usepackage{amsthm}
\usepackage{tikz-cd}
\usepackage{comment}
\usepackage{lipsum}
\usepackage{ifthen}
\usepackage{truthtable} %para las tablas de verdad
\usepackage[x11names, table]{xcolor}

\usepackage{multicol}
\usepackage{paracol}
\usepackage[shortlabels]{enumitem}
\setenumerate[1]{label=\MakeLowercase{\roman*}), ref=\roman*}
\setenumerate[2]{label=\MakeLowercase{\arabic*}), ref=\alph*}

% Para escribir el "tal que" de los conjuntos
\providecommand\st{\;|\;}

%Para el uso de \Set y \Set*
\providecommand\given{}
\newcommand\SetSymbol[1][]{\nonscript\:#1\vert\allowbreak\nonscript\:\mathopen{}}
\DeclarePairedDelimiterX\Set[1]\{\}{\renewcommand\given{\SetSymbol[\delimsize]}#1}
\DeclarePairedDelimiterX\Par[1](){#1}
%Ejercicios
%\newcommand{\pts}[1]{%
  %\ifthenelse{\equal{#1}{1}}{\hfill \textbf{(#1 pt)}}{\hfill\textbf{(#1 pts)}}
%}

\newcounter{Ejer}
\newcounter{Pts}
\setcounter{Ejer}{1}
\setcounter{Pts}{1}
\newcommand{\pts}{}
\newenvironment{ejercicio}[1]{\noindent
    \ifthenelse{\equal{#1}{1}}{\renewcommand{\pts}{\textbf{(#1 pt)}}}{\renewcommand{\pts}{\textbf{(#1 pts)}}}\textbf{Ej. \theEjer} \pts\stepcounter{Ejer}}{\vspace{.3cm}}

%Comandos que utilizamos
\newcommand{\id}{\mathrm{id}}
\newcommand{\op}{{}^{\mathrm{op}}}
\newcommand{\set}[1]{\{#1\}}
\renewcommand{\emptyset}{\varnothing}
\DeclareMathOperator{\ima}{ima}
\DeclareMathOperator{\dom}{dom}

\newcommand{\drawGrid}[2]{%
        \begin{tikzpicture}
            \draw[step=1cm, gray] (0,0) grid (#1,#2);
        \end{tikzpicture}
        }

\begin{document}

    \begin{center}
        {\fontsize{30}{60}\rmfamily \textbf{Tarea 5}} \\ \vspace{.2cm}
        Álgebra Superior 1, 2025-4
    \end{center}
    \begin{flushright}
        \footnotesize \hfill Profesor: Luis Jesús Trucio Cuevas.\\
        \hfill Ayudante: Hugo Víctor García Martínez.
    \end{flushright}

    \noindent\textit{\textbf{Instrucciones.} Resuelve los siguientes ejercicios. Esta tarea es individual y deberá ser entregada presencialmente, durante la clase del \textbf{viernes 8 de agosto}.}\vspace{.4cm}

    \begin{ejercicio}{2}
        Utilizando inducción, demuestra que para todo natural $n \in \mathbb{N}$, se cumple que:
        \[ \sum_{k=0}^{n} k^2 = \frac{n(n+1)(2n+1)}{6} \quad \text{y} \quad \sum_{k=0}^{n} \frac{1}{(k+1)(k+2)} = \frac{n+1}{n+2} \]
    \end{ejercicio}

    \begin{ejercicio}{2}
        Sea $f:\mathbb{R} \to \mathbb{R}$ una función que cumple que para todo real $y$, $f(y)=f(y+2)$. Demuestra que para todo natural $n \in \mathbb{N}$, se tiene que para todo real $x$, $f(x)=f(x-2n)$.
    \end{ejercicio}

    \begin{ejercicio}{1}
        Da dos ejemplos de funciones de $\mathbb{N}$ en $\mathbb{N}$; una que sea sobreyectiva, pero no inyectiva; y otra, que sea inyectiva, pero no sobreyectiva. Demuestra todas tus afirmaciones.
    \end{ejercicio}

    \begin{ejercicio}{2}
        Un natural $m\geq 2$ se dice \textit{compuesto} si existen naturales $a$ y $b$ tales que $1 < a < m$, $1 < b < m$ y $m = ab$; de lo contraro, decimos que $m$ es \textit{primo}. Demuestra que todo natural $n \geq 2$ es producto de números primos.

        \hfill \textit{Hint. Utilice inducción "fuerte".}
    \end{ejercicio}

    \begin{ejercicio}{1}
        ?`Cuantos rectángulos (incluyendo cuadrados) distintos, que téngan sus véretices en una cuadrícula de $n$ por $m$, existen?.
        \begin{center}
            \begin{tikzpicture}
                \draw[RoyalBlue1, fill=RoyalBlue1!30] (0,0) rectangle (2,1);
                \draw[RoyalBlue1, fill=RoyalBlue1!30] (1,2) rectangle (3,5);
                \draw[RoyalBlue1, fill=RoyalBlue1!30] (4,2) rectangle (7,4);
                \draw[RoyalBlue1, fill=RoyalBlue1!30] (7,0) rectangle (8,1);

                \node[gray] at (-.2, -.2) (a) {$0$};
                \draw[step=1cm, gray] (0,0) grid (8,5);
                \foreach \y in {1,...,5} {
                    \node[gray] at (\y, -.3) (a) {$\y$};
                }
                \node[gray] at (6, -.3) (a) {$\dots$};
                \node[gray] at (7, -.3) (a) {$n-1$};
                \node[gray] at (8, -.3) (a) {$n$};
                
                \foreach \y in {1,2,3} {
                    \node[gray] at (-.3, \y) (a) {$\y$};
                }
                \node[gray] at (-.3, 4) (a) {$\vdots$};
                \node[gray] at (-.3, 5) (a) {$m$};
            \end{tikzpicture}
        \end{center}
    \end{ejercicio}    

    \begin{ejercicio}{2}
        Pruebe, sin utilizar el Teorema del Binomio que, para todo $n \in \mathbb{N}$, que $\displaystyle \sum_{k=0}^{n} \binom{n}{k} = 2^n $.
    \end{ejercicio}

\end{document}