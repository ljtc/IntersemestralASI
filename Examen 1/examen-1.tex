\documentclass[letterpaper,DIV=16,headsepline,11pt]{scrartcl}
\usepackage[spanish,mexico]{babel}
\pagenumbering{gobble}
%\setlength{\parindent}{0cm}

\usepackage{mathtools}
\usepackage{amsthm}
\usepackage{ifthen}
\usepackage{unicode-math}
\setmainfont{STIX Two Text}
\setmathfont{STIX Two Math}


\usepackage{multicol}
\usepackage{paracol}
\usepackage[shortlabels]{enumitem}
\setenumerate[1]{label=\MakeLowercase{\roman*}), ref=\roman*}
\setenumerate[2]{label=\MakeLowercase{\arabic*}), ref=\alph*}

% Para escribir el "tal que" de los conjuntos
\providecommand\st{\;|\;}

%Para el uso de \Set y \Set*
\providecommand\given{}
\newcommand\SetSymbol[1][]{\nonscript\:#1\vert\allowbreak\nonscript\:\mathopen{}}
\DeclarePairedDelimiterX\Set[1]\{\}{\renewcommand\given{\SetSymbol[\delimsize]}#1}
\DeclarePairedDelimiterX\Par[1](){#1}
%Ejercicios \newcommand{\pts}[1]{%
  %\ifthenelse{\equal{#1}{1}}{\hfill \textbf{(#1 pt)}}{\hfill\textbf{(#1 pts)}}
%}

\newcounter{Ejer}
\newcounter{Pts}
\setcounter{Ejer}{1}
\setcounter{Pts}{1}
\newcommand{\pts}{}
\newenvironment{ejercicio}[1]{\noindent
    \ifthenelse{\equal{#1}{1} \OR \equal{#1}{+1}}{\renewcommand{\pts}{\textbf{(#1
    pt)}}}{\renewcommand{\pts}{\textbf{(#1 pts)}}}\textbf{Ej. \theEjer}
    \pts\stepcounter{Ejer}}{\vspace{.3cm}}

%Comandos que utilizamos
\newcommand{\id}{\mathrm{id}} \newcommand{\op}{{}^{\mathrm{op}}}
\newcommand{\set}[1]{\{#1\}}
\renewcommand{\emptyset}{\varnothing}
\DeclareMathOperator{\ima}{ima} \DeclareMathOperator{\dom}{dom}

\begin{document}

  \begin{center}
    {\fontsize{30}{60}\rmfamily \textbf{Primer Examen Parcial}} \\
    \vspace{.2cm} Álgebra Superior 1, 2025-4
  \end{center}

  \noindent\textit{\textbf{Instrucciones.} Resuelve los siguientes ejercicios,
  se pueden utilizar libremente resultados vistos en clase, siempre y cuando,
  se indique claramente dónde y cuáles se utilizan.}\vspace{.4cm}

  \begin{ejercicio}{2.5}
    Sean \(I,J,K\) conjuntos no vacíos y supongamos que \(J \cup K = I\). Si 
    \(\{X_i\mid i \in I\}\) es una familia indexada de conjuntos, demuestra que:
    \[ 
    \bigcap_{i \in I} X_i = \Big( \bigcap_{i \in J} X_i \Big) \cap 
    \Big(\bigcap_{i \in K} X_i \Big)
    \]       
  \end{ejercicio}

  \begin{ejercicio}{2.5}
    Demuestra que \((A\times B)\cup(C\times D)\subseteq(A\cup C)\times(B\cup D)\).
    Además da un ejemplo que muestre que la otra contención no siempre se
    cumple.
  \end{ejercicio}

  \begin{ejercicio}{2.5}
    Si \(A\), \(B\) y \(S\) son tales que \(A\cap C=B\cap C\) y 
    \(A\cup C=B\cup C\), entonces \(A=B\).
  \end{ejercicio}

  \begin{ejercicio}{2.5}
    Sean \(A, B\) conjuntos. Demuestra que 
    \(\mathscr{P}(A\cup B)=\mathscr{P}(A)\cap\mathscr{P}(B)\) implica \(A=B\). 
  \end{ejercicio}

  \begin{ejercicio}{+1}
    \textit{Este ejercicio es opcional y sólo se tomará en cuenta si no hay errores en la solución.}

    Sea $A$ un conjunto y $R \subseteq A \times A$ una relación sobre $A$. Demuestre que si $R$ es reflexiva y transitiva, entonces $Q:=R \cup R^{-1}$ es una relación de equivalencia.
  \end{ejercicio}

  %%%%%%%%%%%%%%%%%%%%%%%%%%%%%%%%%%%%%%%%%%%%%%%%%%%%%%%%%%%%
  \vspace{.7cm}
  \setcounter{Ejer}{1}

  \begin{center}
    {\fontsize{30}{60}\rmfamily \textbf{Primer Examen Parcial}} \\
    \vspace{.2cm} Álgebra Superior 1, 2025-4
  \end{center}

  \noindent\textit{\textbf{Instrucciones.} Resuelve los siguientes ejercicios,
  se pueden utilizar libremente resultados vistos en clase, siempre y cuando,
  se indique claramente dónde y cuáles se utilizan.}\vspace{.4cm}

  \begin{ejercicio}{2.5}
    Sean \(I,J,K\) conjuntos no vacíos y supongamos que \(J \cup K = I\). Si 
    \(\{X_i\mid i \in I\}\) es una familia indexada de conjuntos, demuestra que:
    \[ 
    \bigcap_{i \in I} X_i = \Big( \bigcap_{i \in J} X_i \Big) \cap 
    \Big(\bigcap_{i \in K} X_i \Big)
    \]       
  \end{ejercicio}

  \begin{ejercicio}{2.5}
    Demuestra que \((A\times B)\cup(C\times D)\subseteq(A\cup C)\times(B\cup D)\).
    Además da un ejemplo que muestre que la otra contención no siempre se
    cumple.
  \end{ejercicio}

  \begin{ejercicio}{2.5}
    Si \(A\), \(B\) y \(S\) son tales que \(A\cap C=B\cap C\) y 
    \(A\cup C=B\cup C\), entonces \(A=B\). 
  \end{ejercicio}

  \begin{ejercicio}{2.5}
    Sean \(A, B\) conjuntos. Demuestra que 
    \(\mathscr{P}(A\cup B)=\mathscr{P}(A)\cap\mathscr{P}(B)\) implica \(A=B\). 
  \end{ejercicio}

  \begin{ejercicio}{+1}
    \textit{Este ejercicio es opcional y sólo se tomará en cuenta si no hay errores en la solución.}

    Sea $A$ un conjunto y $R \subseteq A \times A$ una relación sobre $A$. Demuestre que si $R$ es reflexiva y transitiva, entonces $Q:=R \cup R^{-1}$ es una relación de equivalencia.
  \end{ejercicio}


\end{document}