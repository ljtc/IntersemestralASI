\documentclass[letterpaper,DIV=14,headsepline,12pt]{scrartcl}
\usepackage[spanish,mexico,shorthands=off,es-lcroman]{babel}
\usepackage{stix2}
\pagenumbering{gobble}
%\setlength{\parindent}{0cm}
\usepackage{mathtools}
\usepackage{amsthm}
\usepackage{tikz-cd}
\usepackage{comment}
\usepackage{lipsum}
\usepackage{ifthen}
%\usepackage{truthtable} %para las tablas de verdad
\usepackage[x11names, table]{xcolor}
\usepackage[colorlinks=true, allcolors=VioletRed2]{hyperref}
\newcommand{\customRef}[2]{\hyperref[#1]{#2~\ref*{#1}}}


\usepackage{multicol}
\usepackage{paracol}
\usepackage[shortlabels]{enumitem}
\setenumerate[1]{label=\MakeLowercase{\roman*}), ref=\roman*}
\setenumerate[2]{label=\MakeLowercase{\arabic*}), ref=\alph*}

% Para escribir el "tal que" de los conjuntos
\providecommand\st{\;|\;}
\providecommand\tq{\;|\;}

%Para el uso de \Set y \Set*
\providecommand\given{}
\newcommand\SetSymbol[1][]{\nonscript\:#1\vert\allowbreak\nonscript\:\mathopen{}}
\DeclarePairedDelimiterX\Set[1]\{\}{\renewcommand\given{\SetSymbol[\delimsize]}#1}
\DeclarePairedDelimiterX\Par[1](){#1}

\newcounter{Ejer}
\newcounter{Pts}
\setcounter{Ejer}{1}
\setcounter{Pts}{1}
\newcommand{\pts}{}
\newenvironment{ejercicio}[1]{\noindent
    \ifthenelse{\equal{#1}{1} \OR \equal{#1}{+1}}{\renewcommand{\pts}{\textbf{(#1 pt)}}}{\renewcommand{\pts}{\textbf{(#1 pts)}}}\textbf{Ej. \theEjer} \pts\stepcounter{Ejer}}{\vspace{.3cm}}

%Comandos que utilizamos
\newcommand{\id}{\mathrm{id}}
\newcommand{\op}{{}^{\mathrm{op}}}
\newcommand{\set}[1]{\{#1\}}
\renewcommand{\emptyset}{\varnothing}
\DeclareMathOperator{\ima}{ima}
\DeclareMathOperator{\dom}{dom}
\newcommand{\quot}[2]{{\raisebox{.2em}{$#1$}\left/\raisebox{-.2em}{$#2$}\right.}}

%Colors
\definecolor{azul}{RGB}{0, 60, 113}
\definecolor{dorado}{RGB}{196, 151, 57}
\definecolor{morado}{RGB}{101, 43, 145}

%Entorno de Demostración y Solución
\renewcommand\qedsymbol{$\blacksquare$}
\makeatletter
    \renewenvironment{proof}[1][]{%
        \par\pushQED{\qed}%
        \normalfont\topsep6pt \partopsep0pt % espacio antes del entorno
        \trivlist
        \item[\hskip\labelsep
                \textbf{\textit{Demostración.}}% Cambiado aquí
        ]#1
        }{%
        \popQED\endtrivlist\@endpefalse
    }
\makeatother

\makeatletter
    \newenvironment{solu}[1][]{%
        \par\pushQED{\hfill \lozenge}%
        \normalfont\topsep6pt \partopsep0pt % espacio antes del entorno
        \trivlist
        \item[\hskip\labelsep
                \textbf{\textit{Solución.}}% Cambiado aquí
        ]#1
        }{%
        \popQED\endtrivlist\@endpefalse
    }
\makeatother

\begin{document}

    \begin{center}
        {\fontsize{30}{60}\rmfamily \textbf{Examen 1 (Solución)}} \\ \vspace{.2cm}
        Álgebra Superior 1, 2025-4
    \end{center}
    \begin{flushright}
        \footnotesize \hfill Profesor: Luis Jesús Trucio Cuevas.\\
        \hfill Ayudante: Hugo Víctor García Martínez.
    \end{flushright}
    
    

    \begin{ejercicio}{2.5}
      Sean \(I,J,K\) conjuntos no vacíos y supongamos que \(J \cup K = I\). Si 
      \(\{X_i\mid i \in I\}\) es una familia indexada de conjuntos, demuestra que:
      \[ 
      \bigcap_{i \in I} X_i = \Big( \bigcap_{i \in J} X_i \Big) \cap 
      \Big(\bigcap_{i \in K} X_i \Big)
      \]       
    \end{ejercicio}
    \begin{proof}
        ($\subseteq$) Supongamos que $x \in \bigcap_{i \in I} X_i$, por definición de intersección (indexada):
        \begin{equation}\label{ej1-inter}
            \forall i \in I (x \in X_i) \equiv \forall ( i \in I \to x \in X_i)
        \end{equation}

        Veamos que $x \in \bigcap_{i \in J} X_i$ y $x \in \bigcap_{i \in K} X_i$. Para lo primero, sea $j \in J$ cualquier elemento, como $I=J \cup K$, entonces $j \in I$ y debido a la \customRef{ej1-inter}{proposición}, $x \in X_j$, por lo tanto $x \in \bigcap_{j \in J} X_j = \bigcap_{i \in J} X_i$. Similarmente, si $k \in K$ es cualquiera, como $I=J \cup K$, entonces $k \in K$ y debido a la \customRef{ej1-inter}{proposición}, $x \in X_k$, por lo tanto $x \in \bigcap_{k \in K} X_k = \bigcap_{i \in K} X_i$. Así que $x \in ( \bigcap_{i \in J} X_i ) \cap ( \bigcap_{i \in K} X_i )$, hemos mostrado que:
        \[ \forall x \bigg( x \in \bigcap_{i \in I} X_i \to x \in \Big( \bigcap_{i \in J} X_i \Big) \cap \Big( \bigcap_{i \in K} X_i \Big) \bigg) \]
        es decir, $\bigcap_{i \in I} X_i \subseteq \Big( \bigcap_{i \in J} X_i \Big) \cap \Big( \bigcap_{i \in K} X_i \Big)$.

        ($\supseteq$) Sea $x \in ( \bigcap_{i \in J} X_i ) \cap ( \bigcap_{i \in K} X_i )$ cualquier elemento, entonces $x \in \bigcap_{i \in J} X_i$ y $x \in \bigcap_{i \in K} X_i$. Luego, por definición de intersección (indexada):
        \begin{equation}\label{ej1-inters}
            \forall ( i \in J \to x \in X_i) \quad \text{y} \quad \forall ( i \in K \to x \in X_i)
        \end{equation}
        
        Ahora, si $i \in I$ es cualquier elemento entonces $i \in J$ o $i \in K$, esto último se debe a que $I=J \cup K$. Por tanto, se sigue de la \customRef{ej1-inters}{proposición}, que (en cualquiera de estos dos casos) $x \in X_i$. Por tanto $x \in \bigcap_{i \in I}  X_i$, probando que:
        \[ \forall x \bigg( x \in \Big( \bigcap_{i \in J} X_i \Big) \cap \Big( \bigcap_{i \in K} X_i \Big) \to x\in \bigcap_{i \in I} X_i \bigg) \]
        es decir, $( \bigcap_{i \in J} X_i ) \cap ( \bigcap_{i \in K} X_i ) \subseteq \bigcap_{i \in I} X_i$.
    \end{proof}

    \begin{ejercicio}{2.5}
      Demuestra que \((A\times B)\cup(C\times D)\subseteq(A\cup C)\times(B\cup D)\).
      Además da un ejemplo que muestre que la otra contención no siempre se cumple.
    \end{ejercicio}
    \begin{proof}
        Veamos primero la contención. Sea $x \in (A \times B) \cup (C \times D)$ cualquier elemento. Entonces $x \in A \times B$, o, $x \in C \times D$ (definición de unión) y existen dos casos.
        \begin{enumerate}
            \item Supongamos que $x \in A \times B$; entonces, existen $a \in A$ y $b \in B$ de modo que $x=(a,b)$ (definición de producto cartesiano). Como $a \in A$, entonces ''$a \in A $ o $a \in C$'' es verdadera, por lo que $a \in A \cup C$. De manera similar, $b \in B \cup D$. Por lo tanto, $x=(a,b) \in (A\cup C) \times (B \cup D)$.
            \item Supongamos que $x \in C \times D$; entonces, existen $c \in C$ y $d \in D$ de modo que $x=(c,d)$ (definición de producto cartesiano). Como $c \in C$, entonces ''$c \in A$ o $c \in C$'' es verdadera, por lo que $c \in A \cup C$. De manera similar, $d \in B \cup D$. Por lo tanto, $x=(c,d) \in (A\cup C) \times (B \cup D)$.
        \end{enumerate}
        En los dos casos anteriores se tiene que $x \in (A \cup C) \times (B \cup D)$; por tanto, hemos probado que:
        \[ \forall x \big( x \in (A\times B)\cup(C\times D) \to x \in (A\cup C)\times(B\cup D) \big) \]
        es decir, $(A\times B)\cup(C\times D)\subseteq(A\cup C)\times(B\cup D)$.
        
        Finalmente, veamos que en general, no se tiene la contención recíproca. Consideremos los conjuntos $A=B=\{0\}$ y $C=D=\{1\}$. De esta manera, por definición de producto cartesiano, $A \times B = \{(0,0)\} $ y $ C \times D = \{(1,1)\}$; de donde:
        \[ (A\times B)\cup(C\times D) = \{(0,0),(1,1)\} \]
        
        Sin embargo, como $1 \in C$, entonces $1 \in A \cup C$; y, como $0 \in B$, entonces $0 \in B \cup D$. Consecuentemente $(1,0) \in (A\cup C)\times(B\cup D)$, pero, $(1,0) \notin (A\times B)\cup(C\times D) = \{(0,0),(1,1)\}$ (de lo contrario $(1,0)=(0,0)$ o $(1,0)=(1,1)$; y en ambos casos $0=1$, lo cual es imposible). Por lo tanto:
        \[ \exists y \big( y \in (A\times B)\cup(C\times D) \land y \notin (A\cup C)\times(B\cup D) \big) \big) \]
        es decir, $(A\cup C)\times(B\cup D) \not\subseteq (A\times B)\cup(C\times D)$.
    \end{proof}

    \begin{ejercicio}{2.5}
      Si \(A\), \(B\) y \(C\) son tales que \(A\cap C=B\cap C\) y 
      \(A\cup C=B\cup C\), entonces \(A=B\).
    \end{ejercicio}
    \begin{proof}
        Sean $A,B,C$ conjuntos, supongamos que:
        \begin{equation}\label{ej3-1}
            A\cap C=B\cap C , \text{ y:}
        \end{equation}
        \begin{equation}\label{ej3-2}
            A\cup C=B\cup C .
        \end{equation}
        
        Probaremos $A=B$ por doble contención.
        
        ($\subseteq$) Sea $x \in A$ arbitrario. Observamos que así, $x \in A \cup C$, y, de la \customRef{ej3-2}{igualdad}, se obtiene que $x \in B \cup C$; es decir, $x \in B$ o $x \in C$. Y hay dos casos.
        \begin{enumerate}
            \item Si $x \in B$, entonces $x \in B$.
            \item Si $x \in C$, dado que $x \in A$, resulta que $x \in A \cap C$. Se sigue de lo anterior (y de \ref{ej3-1}) que $x \in B \cap C$; particularmente, $x \in B$.
        \end{enumerate}
        
        En ambos casos $x \in B$; luego, hemos probado que $\forall x (x \in A \to x \in B)$, equivalentemente, $A \subseteq B$.
        
        ($\supseteq$) Dado que las hipótesis (\customRef{ej3-1}{igualdad} e \customRef{ej3-2}{igualdad}) son simétricas respecto a $A$ y $B$, esta contención es análoga. Por lo tanto, $B \subseteq A$.
        
        Por doble contención, se tiene que $A=B$.
    \end{proof}

    \begin{ejercicio}{2.5}
      Sean \(A, B\) conjuntos. Demuestra que 
      \(\mathscr{P}(A\cup B)=\mathscr{P}(A)\cap\mathscr{P}(B)\) implica \(A=B\). 
    \end{ejercicio}
    \begin{proof}
        Supongamos que $\mathscr{P}(A\cup B)=\mathscr{P}(A)\cap\mathscr{P}(B)$. Nótese que $A \subseteq A \cup B$ (visto en clase). Entonces por definición del conjunto potencia, $A \in \mathscr{P}(A\cup B)$ y se sigue de la hipótesis que $A \in \mathscr{P}(A) \cap \mathscr{P}(B)$; particularmentem $A \in \mathscr{P}(B)$; es decir $A \subseteq B$.
        
        Dado que las hipótesis son simétricas respecto a $A$ y $B$, es análogo al párrafo anterior que $B \subseteq A$. Mostrando, por doble contencion, que $A=B$.
    \end{proof}
    
    \begin{ejercicio}{+1}
      \textit{Este ejercicio es opcional y sólo se tomará en cuenta si no hay 
      errores en la solución.}

      Sean $A$ un conjunto y $R \subseteq A \times A$ una relación sobre $A$.
      Demuestre que si $R$ es reflexiva y transitiva, entonces $Q:=R \cap R^{-1}$
      es una relación de equivalencia.
    \end{ejercicio}
    \begin{proof}
        Supongamos que $R$ es reflexiva y transitiva. Nótese que, como $R \subseteq A \times A$, entonces $R^{-1} \subseteq A \times A$, y así $Q\subseteq A \times A$ (es decir, $Q$ es relación sobre $A$).
       
        ($Q$ es reflexiva) Sea $x \in A$ cualquiera, entonces por ser $R$ reflexiva, $(x,x) \in R$. Por definición de la relación inversa, $(x,x) \in R^{-1}$, y entonces $(x,x) \in R \cap R^{-1} = Q$, es decir:
        \[ \forall x \in A (x \mathrel{Q} x) \quad \text{es decir, $Q$ es reflexiva.} \]
        
        ($Q$ es simétrica) Sean $x,y \in A$ y supongamos que $(x,y) \in Q$. Por definición de $Q$, se tiene que $(x,y) \in R$ y $(x,y) \in R^{-1}$. Como $(x,y) \in R$, entonces $(y,x) \in R^{-1}$; y, como $(x,y) \in R^{-1}$, entonces $(y,x) \in R$. Esto prueba que $(y,x) \in R \cap R^{-1}=Q$. Por tanto:
        \[ \forall x,y \in A (x \mathrel{Q} y \to y \mathrel{Q} x) \quad \text{es decir, $Q$ es simétrica.}\]
        
        ($Q$ es transitiva) Sean $u,v,w \in A$ y supongamos que $(u,v) \in Q$ y $(v,w) \in Q$. Por definición de $Q$, resulta que $(u,v) \in R$, $(u,v) \in R^{-1}$, $(v,w) \in R$ y $(v,w) \in R^{-1}$. Dado que $(u,v) \in R$, $(v,w) \in R$ y $R$ es transitiva, entonces $(u,w) \in R$. Además, como $(u,v) \in R^{-1}$ y $(v,w) \in R^{-1}$, por definición de relación inversa, se tiene $(v,u) \in R$ y $(w,v) \in R$. De nuevo, por transitividad de $R$, se da $(w,u) \in R$; esto es, $(u,w) \in R^{-1}$. Por lo tanto $(u,w) \in R \cap R^{-1}=Q$, probando que:
        \[ \forall u,v,w \in A \big( (u \mathrel{Q} v \land v \mathrel{Q} w ) \to u \mathrel{Q} w \big) \quad \text{es decir, $Q$ es transitiva.} \]
        
        Así que $Q$ es de equivalencia.
    \end{proof}

\end{document}
