\documentclass[letterpaper,DIV=16,headsepline,11pt]{scrartcl}
\usepackage[spanish,mexico]{babel}
\pagenumbering{gobble}
%\setlength{\parindent}{0cm}

\usepackage{mathtools}
\usepackage{amsthm}
\usepackage{ifthen}
\usepackage{unicode-math}
\setmainfont{STIX Two Text}
\setmathfont{STIX Two Math}

\usepackage{multicol}
\usepackage{paracol}
\usepackage[shortlabels]{enumitem}
\setenumerate[1]{label=\MakeLowercase{\roman*}), ref=\roman*}
\setenumerate[2]{label=\MakeLowercase{\arabic*}), ref=\alph*}

\ExplSyntaxOn
\box_new:N \l_repetir_caja
\NewDocumentEnvironment{Repeat}{m +b}{
  \box_clear:N \l_repetir_caja
  \vbox_set:Nn \l_repetir_caja {#2}
  \prg_replicate:nn {#1} {\box_use:N \l_repetir_caja }
}{}
\ExplSyntaxOff

%Para el uso de \Set y \Set*
\providecommand\given{}
\newcommand\SetSymbol[1][]{\nonscript\:#1\vert\allowbreak\nonscript\:\mathopen{}}
\DeclarePairedDelimiterX\Set[1]\{\}{\renewcommand\given{\SetSymbol[\delimsize]}#1}
\DeclarePairedDelimiterX\Par[1](){#1}
%Ejercicios \newcommand{\pts}[1]{%
  %\ifthenelse{\equal{#1}{1}}{\hfill \textbf{(#1 pt)}}{\hfill\textbf{(#1 pts)}}
%}

\newcounter{Ejer}
\newcounter{Pts}
\setcounter{Ejer}{1}
\setcounter{Pts}{1}
\newcommand{\pts}{}
\newenvironment{ejercicio}[1]{\noindent
    \ifthenelse{\equal{#1}{1} \OR \equal{#1}{+1}}{\renewcommand{\pts}{\textbf{(#1
    pt)}}}{\renewcommand{\pts}{\textbf{(#1 pts)}}}\textbf{Ej. \theEjer}
    \pts\stepcounter{Ejer}}{\vspace{.3cm}}

%Comandos que utilizamos
\newcommand{\id}{\mathrm{id}} \newcommand{\op}{{}^{\mathrm{op}}}
\newcommand{\set}[1]{\{#1\}}
\renewcommand{\emptyset}{\varnothing}
\DeclareMathOperator{\ima}{ima} \DeclareMathOperator{\dom}{dom}

\begin{document}

\begin{Repeat}{2}
\begin{center}
  {\huge\rmfamily \textbf{Reposición del Primer Examen Parcial}} \\
  \vspace{.2cm} Álgebra Superior 1, 2025-4
\end{center}
\begin{flushright}
  \footnotesize \hfill Profesor: Luis Jesús Trucio Cuevas.\\
  \hfill Ayudante: Hugo Víctor García Martínez.
\end{flushright}

\noindent\textit{\textbf{Instrucciones.} Resuelve los siguientes ejercicios,
se pueden utilizar libremente resultados vistos en clase, siempre y cuando,
se indique claramente dónde y cuáles se utilizan.}\vspace{.4cm}

\begin{ejercicio}{2.5}
  Sean \(A\) un conjunto y \(\Set{B_i\given i\in I}\), con \(I\ne\emptyset\), una
  familia de conjuntos. Demuestra que 
  \(A\subseteq\bigcap_{i \in I} B_i\) si y sólo si 
  \(\forall i \in I\; (A\subseteq B_i)\).
\end{ejercicio}

\begin{ejercicio}{2.5}
  Sean \(A\) y \(B\) conjuntos. Demuestra que \(A\setminus B=A\setminus(A\cap B)\).
\end{ejercicio}

\begin{ejercicio}{2.5}
  Sean \(A\) y \(B\) dos conjuntos. Demuestra que \(A\times B=\bigcup_{b \in B} A\times \Set{b}\).
\end{ejercicio}

\begin{ejercicio}{2.5}
  Sean \(A\) y \(B\) conjuntos. Demuestra que 
  \(\mathscr{P}(A\cup B)=\mathscr{P}(A)\cap\mathscr{P}(B)\) implica \(A=B\).
\end{ejercicio}

\vspace{2cm}
\end{Repeat}

\end{document}