	\documentclass[letterpaper,DIV=14,headsepline,12pt]{scrartcl}
\usepackage[spanish,mexico,shorthands=off,es-lcroman]{babel}
\usepackage{stix2}
\pagenumbering{gobble}
%\setlength{\parindent}{0cm}

\usepackage{mathtools}
\usepackage{amsthm}
\usepackage{tikz-cd}
\usepackage{comment}
\usepackage{lipsum}
\usepackage{ifthen}
\usepackage{truthtable} %para las tablas de verdad

\usepackage{multicol}
\usepackage{paracol}
\usepackage[shortlabels]{enumitem}
\setenumerate[1]{label=\MakeLowercase{\roman*}), ref=\roman*}
\setenumerate[2]{label=\MakeLowercase{\arabic*}), ref=\alph*}

% Para escribir el "tal que" de los conjuntos
\providecommand\st{\;|\;}
\providecommand\tq{\;|\;}

%Para el uso de \Set y \Set*
\providecommand\given{}
\newcommand\SetSymbol[1][]{\nonscript\:#1\vert\allowbreak\nonscript\:\mathopen{}}
\DeclarePairedDelimiterX\Set[1]\{\}{\renewcommand\given{\SetSymbol[\delimsize]}#1}
\DeclarePairedDelimiterX\Par[1](){#1}

\newcounter{Ejer}
\newcounter{Pts}
\setcounter{Ejer}{1}
\setcounter{Pts}{1}
\newcommand{\pts}{}
\newenvironment{ejercicio}[1]{\noindent
    \ifthenelse{\equal{#1}{1}}{\renewcommand{\pts}{\textbf{(#1 pt)}}}{\renewcommand{\pts}{\textbf{(#1 pts)}}}\textbf{Ej. \theEjer} \pts\stepcounter{Ejer}}{\vspace{.3cm}}

%Comandos que utilizamos
\newcommand{\id}{\mathrm{id}}
\newcommand{\op}{{}^{\mathrm{op}}}
\newcommand{\set}[1]{\{#1\}}
\renewcommand{\emptyset}{\varnothing}
\DeclareMathOperator{\ima}{ima}
\DeclareMathOperator{\dom}{dom}

\begin{document}

    \begin{center}
        {\fontsize{30}{60}\rmfamily \textbf{Tarea 1}} \\ \vspace{.2cm}
        Álgebra Superior 1, 2025-4
    \end{center}
    \begin{flushright}
        \footnotesize \hfill Profesor: Luis Jesús Trucio Cuevas.\\
        \hfill Ayudante: Hugo Víctor García Martínez.
    \end{flushright}

    \noindent\textit{\textbf{Instrucciones.} Resuelve los siguientes ejercicios. Esta tarea es individual y deberá ser entregada presencialmente, durante la clase del \textbf{lunes 16 de junio}.}\vspace{.4cm}

    \begin{ejercicio}{1}
        Demuestra las siguientes equivalencias lógicas.
        \begin{multicols}{2}
            \begin{enumerate}
                \item \(\alpha\land\alpha\equiv\alpha\).
                \item \(\alpha\land(\alpha\lor\beta)\equiv\alpha\).
                \item \(\alpha\lor(\alpha\land\beta)\equiv\alpha\).
                \item \(\alpha\lor(\beta\land\gamma)\equiv(\alpha\lor\beta)\land(\alpha\lor\gamma)\).
                \item \(\neg(\alpha\land\beta)\equiv\neg\alpha\lor\neg\beta\).
                \item \(\neg\neg\alpha\equiv\alpha\).
            \end{enumerate}
        \end{multicols}
    \end{ejercicio}

    \begin{ejercicio}{1}
        Escribe fórmulas lógicas (de primer orden) que, a tu criterio, capturen mejor cada una de las siguientes afirmaciones.
        \columnratio{0.35}
        \begin{paracol}{2}
            \begin{enumerate}
                \item Cada persona viva respira.
                \item $2$ es el único primo par.
                \item Existe un hombre inmortal.
            \end{enumerate}
            \switchcolumn
             \begin{enumerate}\setcounter{enumi}{3}
                \item No existen estudiantes en Ciudad Universitaria que sean felices.
                \item Todos los peces del acuario de la facultad se aparean algún un individuo.
            \end{enumerate}
        \end{paracol}
    \end{ejercicio}

    \begin{ejercicio}{1}
        Escribe la negación de las siguientes proposiciones. Si el inciso está en español, da tu respuesta también en español.
        \columnratio{0.22,0.42}
        \begin{paracol}{3}
            \begin{enumerate}
                \item $\alpha \leftrightarrow \beta$.
                \item $\lnot \alpha \to \gamma$.
                \item $\gamma \to (\delta \to \gamma)$.
            \end{enumerate}

            \switchcolumn
             \begin{enumerate}\setcounter{enumi}{3}
                \item $\exists x (P(x) \land (Q(x) \land S(x)))$.
                \item $\forall a (P(a) \to \exists b (R(a,b)))$.
                \item $\exists b \forall x (\forall y(P(y)) \leftrightarrow (Q(x,y) \land R(b)))$
            \end{enumerate}

            \switchcolumn
             \begin{enumerate}\setcounter{enumi}{6}
                \item Si $x$ es un número primo y es mayor que 4, $n$ es impar.
                \item Hay cierto elemento de $A$ que es real, pero no real.
            \end{enumerate}
        \end{paracol}
    \end{ejercicio}

    \begin{ejercicio}{1}
        Indica cuáles de las siguientes proposiciones son tautologías o contradicciones. Para aquellas que no sean ninguna de las dos, da una equivalencia lógica que utilicen únicamente los conectivos negación y disyunción.
        \begin{multicols}{3}
            \begin{enumerate}
                \item $\lnot (\gamma \land \gamma)$.
                \item $\alpha \to \alpha$.
                \item $\alpha \land (\alpha \lor \beta)$.
                \item $\alpha \lor (\alpha \land \beta)$.
                \item $\beta \to (\alpha \to \beta)$.
                \item $\Par{\lnot \gamma \land \Par{(\lnot \gamma) \lor \beta}} \leftrightarrow S$.
                \item $\lnot \delta \land \delta$.
                \item $(\gamma \to \eta) \to (\lnot \eta \to \lnot \gamma)$.
                \item $\beta \land \alpha$.
            \end{enumerate}
        \end{multicols}
    \end{ejercicio}

    \begin{ejercicio}{1}
        Traduce \textit{solamente cuatro} equivalencias lógicas del \textbf{Ejercicio 1} a igualdades entre conjuntos. Posteriormente, demuestra tales igualdades.
    \end{ejercicio}

    \begin{ejercicio}{1}
        Denotamos por $A \vartriangle B$ a la diferencia simétrica entre los conjuntos $A$ y $B$. Demuestra las siguientes propiedades de esta operación.
        \begin{multicols}{2}
            \begin{enumerate}
                \item $A \vartriangle \emptyset = A$.
                \item $A \cup B = (A \vartriangle B) \vartriangle (A \cap B)$.
            \end{enumerate}
        \end{multicols}
    \end{ejercicio}

    \newpage
    \begin{ejercicio}{1}
        Sea $A:=\{x \in \mathbb{R} : \text{$x > 0$} \}$. Para cada $y \in \mathbb{R}$ se define el conjunto $B(y)$ (pues depende de $y$), como $B(y):=\{x \in \mathbb{R} : |x-y|<0.15 \}$. Utilizando operaciones de conjuntos, escribe en términos de los conjuntos anteriores, la colección cuyos elementos sean:
        \begin{enumerate}
            \item Todos los enteros menores o iguales a $0$.
            \item Los reales negativos mayores a $-0.15$.
            \item Todos los irracionales cuya distancia a $2$ es mayor o igual a $0.15$.
            \item Todos los racionales que distan de algún entero en menos de $0.15$.
        \end{enumerate}
    \end{ejercicio}

\end{document}