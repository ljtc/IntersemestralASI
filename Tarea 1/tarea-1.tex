	\documentclass[letterpaper,DIV=14,headsepline,12pt]{scrartcl}
\usepackage[spanish,mexico,shorthands=off,es-lcroman]{babel}
\usepackage{stix2}
\pagenumbering{gobble}
%\setlength{\parindent}{0cm}
\usepackage{mathtools}
\usepackage{amsthm}
\usepackage{tikz-cd}
\usepackage{comment}
\usepackage{lipsum}
\usepackage{ifthen}
\usepackage{truthtable} %para las tablas de verdad

\usepackage{multicol}
\usepackage{paracol}
\usepackage[shortlabels]{enumitem}
\setenumerate[1]{label=\MakeLowercase{\roman*}), ref=\roman*}
\setenumerate[2]{label=\MakeLowercase{\arabic*}), ref=\alph*}

% Para escribir el "tal que" de los conjuntos
\providecommand\st{\;|\;}
\providecommand\tq{\;|\;}

%Para el uso de \Set y \Set*
\providecommand\given{}
\newcommand\SetSymbol[1][]{\nonscript\:#1\vert\allowbreak\nonscript\:\mathopen{}}
\DeclarePairedDelimiterX\Set[1]\{\}{\renewcommand\given{\SetSymbol[\delimsize]}#1}
\DeclarePairedDelimiterX\Par[1](){#1}

\newcounter{Ejer}
\newcounter{Pts}
\setcounter{Ejer}{1}
\setcounter{Pts}{1}
\newcommand{\pts}{}
\newenvironment{ejercicio}[1]{\noindent
    \ifthenelse{\equal{#1}{1} \OR \equal{#1}{+1}}{\renewcommand{\pts}{\textbf{(#1 pt)}}}{\renewcommand{\pts}{\textbf{(#1 pts)}}}\textbf{Ej. \theEjer} \pts\stepcounter{Ejer}}{\vspace{.3cm}}

%Comandos que utilizamos
\newcommand{\id}{\mathrm{id}}
\newcommand{\op}{{}^{\mathrm{op}}}
\newcommand{\set}[1]{\{#1\}}
\renewcommand{\emptyset}{\varnothing}
\DeclareMathOperator{\ima}{ima}
\DeclareMathOperator{\dom}{dom}

\begin{document}

    \begin{center}
        {\fontsize{30}{60}\rmfamily \textbf{Tarea 1}} \\ \vspace{.2cm}
        Álgebra Superior 1, 2025-4
    \end{center}
    \begin{flushright}
        \footnotesize \hfill Profesor: Luis Jesús Trucio Cuevas.\\
        \hfill Ayudante: Hugo Víctor García Martínez.
    \end{flushright}

    \noindent\textit{\textbf{Instrucciones.} Resuelve los siguientes ejercicios. Esta tarea es individual y deberá ser entregada presencialmente, durante la clase del \textbf{lunes 16 de junio}.}\vspace{.4cm}

    \begin{ejercicio}{1}
        Demuestra las siguientes equivalencias lógicas.
        \begin{multicols}{2}
            \begin{enumerate}
                \item \(\alpha\land(\alpha\lor\beta)\equiv\alpha\).
                \item \(\alpha\lor(\alpha\land\beta)\equiv\alpha\).
                \item \(\alpha\lor(\beta\land\gamma)\equiv(\alpha\lor\beta)\land(\alpha\lor\gamma)\).
                \item \(\neg(\alpha\land\beta)\equiv\neg\alpha\lor\neg\beta\).
            \end{enumerate}
        \end{multicols}
    \end{ejercicio}

    \begin{ejercicio}{1}
        Escribe fórmulas lógicas (de primer orden) que, a tu criterio, capturen mejor cada una de las siguientes afirmaciones.
        \columnratio{0.35}
        \begin{paracol}{2}
            \begin{enumerate}
                \item Cada persona viva respira.
                \item $2$ es el único primo par.
                \item Existe un hombre inmortal.
            \end{enumerate}
            \switchcolumn
             \begin{enumerate}\setcounter{enumi}{3}
                \item No existen estudiantes en Ciudad Universitaria que sean felices.
                \item Todos los peces del acuario de la facultad se aparean con otro pez.
            \end{enumerate}
        \end{paracol}
    \end{ejercicio}

    \begin{ejercicio}{1}
        Escribe la negación de las siguientes proposiciones. Si el inciso está en español, escribe tu respuesta en español.
        \columnratio{0.21,0.42}
        \begin{paracol}{3}
            \begin{enumerate}
                \item $\alpha \leftrightarrow \beta$.
                \item $\lnot \alpha \to \gamma$.
                \item $\gamma \to (\delta \to \gamma)$.
            \end{enumerate}

            \switchcolumn
             \begin{enumerate}\setcounter{enumi}{3}
                \item $\exists x \big( \alpha(x) \land \big( \beta(x) \land \gamma(x) \big) \big)$.
                \item $\forall a \big(\alpha(a) \to \exists b \big( \beta(a,b) \big) \big)$.
                \item $\exists b \forall x \big( \forall y(\alpha(y)) \leftrightarrow \big( Q(x,y) \land R(b) \big) \big)$.
            \end{enumerate}

            \switchcolumn
             \begin{enumerate}\setcounter{enumi}{6}
                \item Si $n$ es un número primo y es mayor que 4, $n$ es impar.
                \item Hay cierto elemento de $A$ que es real, pero no entero.
            \end{enumerate}
        \end{paracol}
    \end{ejercicio}

    \begin{ejercicio}{1}
        Indica cuáles de las siguientes proposiciones son tautologías o contradicciones. Para aquellas que sean contingentes, da una equivalencia lógica que utilice únicamente los conectivos negación y disyunción. No es necesario justificar.
        \begin{multicols}{3}
            \begin{enumerate}
                \item $\lnot (\gamma \land \gamma)$.
                \item $\alpha \to \alpha$.
                \item $\alpha \land (\alpha \lor \beta)$.
                \item $\alpha \lor (\alpha \land \beta)$.
                \item $\beta \to (\alpha \to \beta)$.
                \item $\big( \lnot \gamma \land (\lnot \gamma \lor \beta) \big) \leftrightarrow \gamma$.
                \item $\lnot \delta \leftrightarrow \delta$.
                \item $(\gamma \to \eta) \to (\lnot \eta \to \lnot \gamma)$.
                \item $\beta \land \alpha$.
            \end{enumerate}
        \end{multicols}
    \end{ejercicio}

    \begin{ejercicio}{1}
        Traduce las siguientes equivalencias lógicas a igualdades entre conjuntos. Demuestra las igualdades que propusiste.
        \begin{multicols}{2}
            \begin{enumerate}
                \item \(\alpha\lor(\beta\land\gamma)\equiv(\alpha\lor\beta)\land(\alpha\lor\gamma)\).
                \item \(\neg(\alpha\land\beta)\equiv\neg\alpha\lor\neg\beta\).
            \end{enumerate}
        \end{multicols}
    \end{ejercicio}

    \begin{ejercicio}{1}
        Sean $A$ y $X$ conjuntos de modo que $A \subseteq X$. Demuestra \textit{un inciso} de cada una de las siguientes columnas (tres igualdades en total).
        \begin{multicols}{3}
            \begin{enumerate}
                \item $A \cap \emptyset = \emptyset$.
                \item $A \cup \emptyset = A$.
                \item $A \cup A = A$.
                \item $A \cap X = A$.
                \item $A \cup X = X$.
                \item $A \cap A = A$.
                \item $A \cap (X \setminus A) = \emptyset$.
                \item $A \cup (X \setminus A) = X$.
                \item $X \setminus (X \setminus A) = A$.
            \end{enumerate}
        \end{multicols}
    \end{ejercicio}

    \begin{ejercicio}{1}
        Denotamos por $A \vartriangle B$ a la diferencia simétrica entre los conjuntos $A$ y $B$. Demuestra  que $A \cup B = (A \vartriangle B) \vartriangle (A \cap B)$.
    \end{ejercicio}

    \begin{ejercicio}{1}
        Sean $A$ y $B$ conjuntos. Demuestra que:
        \begin{enumerate}
            \item $A \subseteq A \cap B$ si y sólo si $A \subseteq B$.
            \item $A \cup B \subseteq B$ si y sólo si $A \subseteq B$.
        \end{enumerate}
    \end{ejercicio}

    \begin{ejercicio}{1}
        Sean $A$ y $B$ conjuntos. Prueba que $A \subseteq B$ si y sólo si $\mathscr{P}(A) \subseteq \mathscr{P}(B)$.
    \end{ejercicio}

    \begin{ejercicio}{1}
        Muestra que, en general, \textit{no se da} la igualdad $\mathscr{P}(A \setminus B) = \mathscr{P}(A) \setminus \mathscr{P}(B)$.
    \end{ejercicio}
\end{document}