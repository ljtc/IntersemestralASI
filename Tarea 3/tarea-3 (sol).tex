\documentclass[letterpaper,DIV=14,headsepline,12pt]{scrartcl}
\usepackage[spanish,mexico,shorthands=off,es-lcroman]{babel}
\usepackage{stix2}
\pagenumbering{gobble}
%\setlength{\parindent}{0cm}
\usepackage{mathtools}
\usepackage{amsthm}
\usepackage{tikz-cd}
\usepackage{comment}
\usepackage{lipsum}
\usepackage{ifthen}
\usepackage{truthtable} %para las tablas de verdad
\usepackage[table]{xcolor}

\usepackage{multicol}
\usepackage{paracol}
\usepackage[shortlabels]{enumitem}
\setenumerate[1]{label=\MakeLowercase{\roman*}), ref=\roman*}
\setenumerate[2]{label=\MakeLowercase{\arabic*}), ref=\alph*}

% Para escribir el "tal que" de los conjuntos
\providecommand\st{\;|\;}
\providecommand\tq{\;|\;}

%Para el uso de \Set y \Set*
\providecommand\given{}
\newcommand\SetSymbol[1][]{\nonscript\:#1\vert\allowbreak\nonscript\:\mathopen{}}
\DeclarePairedDelimiterX\Set[1]\{\}{\renewcommand\given{\SetSymbol[\delimsize]}#1}
\DeclarePairedDelimiterX\Par[1](){#1}

\newcounter{Ejer}
\newcounter{Pts}
\setcounter{Ejer}{1}
\setcounter{Pts}{1}
\newcommand{\pts}{}
\newenvironment{ejercicio}[1]{\noindent
    \ifthenelse{\equal{#1}{1} \OR \equal{#1}{+1}}{\renewcommand{\pts}{\textbf{(#1 pt)}}}{\renewcommand{\pts}{\textbf{(#1 pts)}}}\textbf{Ej. \theEjer} \pts\stepcounter{Ejer}}{\vspace{.3cm}}

%Comandos que utilizamos
\newcommand{\id}{\mathrm{id}}
\newcommand{\op}{{}^{\mathrm{op}}}
\newcommand{\set}[1]{\{#1\}}
\renewcommand{\emptyset}{\varnothing}
\DeclareMathOperator{\ima}{ima}
\DeclareMathOperator{\dom}{dom}
\newcommand{\quot}[2]{{\raisebox{.2em}{$#1$}\left/\raisebox{-.2em}{$#2$}\right.}}

%Colors
\definecolor{azul}{RGB}{0, 60, 113}
\definecolor{dorado}{RGB}{196, 151, 57}
\definecolor{morado}{RGB}{101, 43, 145}

%Entorno de Demostración y Solución
\renewcommand\qedsymbol{$\blacksquare$}
\makeatletter
    \renewenvironment{proof}[1][]{%
        \par\pushQED{\qed}%
        \normalfont\topsep6pt \partopsep0pt % espacio antes del entorno
        \trivlist
        \item[\hskip\labelsep
                \textbf{\textit{Demostración.}}% Cambiado aquí
        ]#1
        }{%
        \popQED\endtrivlist\@endpefalse
    }
\makeatother

\makeatletter
    \newenvironment{solucion}[1][]{%
        \par\pushQED{\hfill \lozenge}%
        \normalfont\topsep6pt \partopsep0pt % espacio antes del entorno
        \trivlist
        \item[\hskip\labelsep
                \textbf{\textit{Solución.}}% Cambiado aquí
        ]#1
        }{%
        \popQED\endtrivlist\@endpefalse
    }
\makeatother

\begin{document}

    \begin{center}
        {\fontsize{30}{60}\rmfamily \textbf{Tarea 3 (Solución)}} \\ \vspace{.2cm}
        Álgebra Superior 1, 2025-4
    \end{center}
    \begin{flushright}
        \footnotesize \hfill Profesor: Luis Jesús Trucio Cuevas.\\
        \hfill Ayudante: Hugo Víctor García Martínez.
    \end{flushright}

    \begin{ejercicio}{2.5}
        Se dice que una función $h:X \to Y$ es \textbf{constante} si y sólo si para cualesquiera $x,y \in X$ se tiene que $h(x)=h(y)$.
        \begin{enumerate}
            \item Demuestra que la composición de funciones constantes es una función constante.
            \item Encuentra dos funciones \textit{no} constantes; $f:\mathbb{N} \to \mathbb{Z}$ y $g:\mathbb{Z}\to \mathbb{N}$, cuya composición (la función $g \circ f:\mathbb{N} \to \mathbb{N}$) sea constante.
        \end{enumerate}
    \end{ejercicio}

   \begin{solucion}
        (i) Supongamos que $f:A \to B$ y $g:B \to C$ son funciones constantes. Como $g \circ f$ tiene dominio $A$ y codominio $C$, habremos de verificar que:
        \[ \forall x,y \in A \big( \big(g \circ f\big)(x) = \big(g \circ f\big)(y) \big) \]

        En efecto, sean $x,y \in A$ cualesquiera, entonces:
        \begin{align*}
            \big(g \circ f\big) (x) = g\big(f (x) \big) \tag*{Def. de composición} \\
            = g\big(f (y) \big) \tag*{$f$ es constante} \\
            = \big(g \circ f\big) (y) \tag*{Def. de composición} \\
        \end{align*}
        por lo que $g \circ f$ es constante.

        (ii) Sea $f:\mathbb{N} \to \mathbb{Z}$ definida como $f(n)=0$ si $n$ es par; y, $f(n)=1$ si $n$ es impar. Y definimos $g:\mathbb{Z} \to \mathbb{N}$ como $g(z)=z(z-1)$.

        Notemos que $f$ no es constante, ya que $f(2)=0$, $f(1)=1$ y $2 \neq 1$; y, $g$ tampoco es constante, ya que $g(2)=2$, $g(3)=6$ y $2 \neq 3$. Sin embargo, para cada $n \in \mathbb{N}$ se tiene que:
        \begin{enumerate}[a)]
            \item Si $n$ es par, entonces $f(n)=0$ y $g\big( f(n)\big)=g(0)=0$.
            \item Si $n$ es impar, entonces $f(n)=0$ y $g\big( f(n)\big)=g(1)=0$.
        \end{enumerate}
        
        Por lo tanto, para cualesquiera $x,y \in \mathbb{N}$, $\big(g \circ f\big) (x)=0=\big(g \circ f\big) (y)$, probando que $g \circ f$ es constante.
    \end{solucion}

    \begin{ejercicio}{2.5}
        En cada inciso, determina si la correspondiente función es inyectiva, sobreyectiva, o biyectiva. Demuestra la conclusión a la que llegaste (es decir, prueba si la función tiene o no la propiedad que se afrima).
        \begin{enumerate}
            \item $h:\{0,1,2\} \to \{x,y\}$ definida como $h=\{(0,x),(1,x),(2,y)\}$, aquí $x \neq y$.
            \item $g:\mathbb{R}\to \mathbb{R}$ dada, para cada $x \in \mathbb{R}$, por $g(x)=4x+55$.
            \item $f:\mathbb{N}\to \mathscr{P}(\mathbb{N})$ dada, para cada $n \in \mathbb{N}$, por $f(n)=\{0,n\}$.
        \end{enumerate}
    \end{ejercicio}

    \begin{solucion}
        (i) Notese que $x \in \ima(h)$ ya que $x=h(0)$; y, $y \in \ima(h)$ pues $y=h(1)$. Por lo tanto $\{x,y\} \subseteq \ima(h)$, y como $\{x,y\}$ es codominio de $h$, $\ima(h) \subseteq \{x,y\}$. Así que $\ima(h)=\{x,y\}$, esto es, $h$ \textbf{es sobreyectiva}.

        Pero $h$ \textbf{no es inyectiva}, pues $h(0)=h(1)=x$ y $0\neq 1$. Así, $h$ \textbf{tampoco es biyectiva}.

        (ii) Esta funcion es biyectiva, en efecto, si $x,y \in \mathbb{R}$ son tales que $g(x)=g(y)$, entonces $4x+55=4y+55$, de donde $4x=4y$ y con ello $x=y$; esto es, $g$ \textbf{es inyectiva}.

        Por otra parte, si $r \in \mathbb{R}$ es cualquiera, $t:=\frac{r-55}{4}$ es un elemento de $\dom(g)=\mathbb{R}$ que satisface:
        \[ g(t)=g\Big(\frac{r-55}{4}\Big) = 4 \Big(\frac{r-55}{4}\Big) +55 = (r-55)+55=r \]
        lo cual prueba, por la arbitrariedad de $r$, que $g$ \textbf{es sobreyectiva}. Así, $g$ \textbf{es biyectiva}.

        (iii) Por último, $f$ no es sorbeyectiva. Notemos que para cada $n \in \mathbb{N}$ se cumple que $0 \in f(n)=\{0,n\}$, esto implica que $f(n)\neq \emptyset$, así que $\forall n \in f(n) \neq \emptyset$, lo cual muestra que $\emptyset$ no está en la imagen de $f$. Pero $\emptyset$ es un eleemento de $\mathscr{P}(\mathbb{N})$, ya que $\emptyset \subseteq \mathbb{N}$, de forma que $\ima(f) \neq \mathscr{P}(\mathbb{N})$, mostrando que $f$ \textbf{no es sobreyectiva}, así, \textbf{tampoco es biyectiva}.

        Ahora, $f$ sí es inyectiva. En efecto, sean $n,m\in \mathbb{N}$ y supongamos que $f(n)=f(m)$, entonces $\{0,n\}=\{0,m\}$\footnote{Como esta igualdad \textit{no es de pares ordenados} no se puede concluir directo que $0=0$ y $n=m$, hay que hacer algo mas.}, como $n \in \{0,m\}$ entonces $n=0$ o $n=m$ (por definición de par) y se tienen dos casos:
        \begin{enumerate}
            \item Si $n=0$, $\{0,n\}=\{0,0\}=\{0\}$; pero por hipótesis, $\{0,n\}=\{0,m\}$; asi que $\{0,m\}=\{0\}$. Como $m \in \{0,m\}=\{0\}$, es necesario que $m=0$. En este caso $n=m$ (ambos son $0$).
            \item Si $n=m$, entonces en este caso $n=m$.
        \end{enumerate}
        En cualquier caso $n=m$, mostrando que $\forall n,m \in \mathbb{N} \big( f(n)=f(m) \rightarrow n=m \big)$, es decir $f$ \textbf{es inyectiva}.
    \end{solucion}

    \begin{ejercicio}{2.5}
        Sean $A$ y $B$ conjuntos arbitrarios y $f:A \to B$. Demuestra que las siguientes proposiciones son equivalentes:
        \begin{enumerate}
            \item $\forall X \subseteq A \; \forall Y \subseteq A \; \big( f[X \setminus Y] \subseteq f[X] \setminus f[Y] \big)$.
            \item $\forall U \subseteq A \; \forall V \subseteq A \; \big( f[U] \cap f[V] \subseteq f[U \cap V] \big)$.
        \end{enumerate}
    \end{ejercicio}
    \begin{proof}
        (i) $\to$ (ii) Supongamos que $\forall X \subseteq A \; \forall Y \subseteq A \; \big( f[X \setminus Y] \subseteq f[X] \setminus f[Y] \big)$. Veamos que $\forall U \subseteq A \; \forall V \subseteq A \; \big( f[U] \cap f[V] \subseteq f[U \cap V] \big)$
       
        Sean $U,V$ subconjuntos de $A$, veamos que ocurre la contención $f[U] \cap f[V] \subseteq f[U \cap V]$. Efectivamente, sea $l\in f[U] \cap f[V]$ cualquier elemento. Así, existen $u \in U$ y $v \in V$ tales que $l=f(u)$ y $l=f(v)$. Como $\{u\},\{v\} \subseteq A$, por hipótesis se tiene que:
        \[ f[\{u\} \setminus \{v\}]\subseteq f[\{u\}] \setminus f[\{v\}] \]
        pero $f[\{u\}]\setminus f[\{v\}] = \{l\} \setminus \{l\} = \emptyset$, por lo que $f[\{u\} \setminus \{v\}]=\emptyset$ y así $\{u\} \setminus \{v\}=\emptyset$, luego $\{u\} \subseteq \{v\}$ y por lo tanto $u=v$. Esto prueba que $u \in V$, así que $u \in U \cap V$ y como $l=f(u)$, entonces $l \in f[U \cap V]$. Por lo que $f[U] \cap f[V] \subseteq f[U \cap V]$.

        (ii) $\rightarrow$ (i) Supongamos que $\forall U \subseteq A \; \forall V \subseteq A \; \big( f[U] \cap f[V] \subseteq f[U \cap V] \big)$, veamos que se satisface $\forall X \subseteq A \; \forall Y \subseteq A \; \big( f[X \setminus Y] \subseteq f[X] \setminus f[Y] \big)$.

        Supongamos que $X,Y \subseteq A$, verifiquemos que $f[X \setminus Y] \subseteq f[X] \setminus f[Y]$. Sea $l \in f[X \setminus Y]$ cualquiera, entonces existe $x \in X \setminus Y$ tal que $l=f(x)$. Esto muestra que $l \in f[X]$ (ya que $x \in X$), resta ver entonces que $l \notin f[Y]$.

        Por contradicción, supongamos que $l \in f[Y]$, entonces existe $y \in Y$ tal que $l=f(y)$, pero, como $\{x\},\{y\} \subseteq A$, aplicando la hipótesis se tiene que:
        \[ f[\{x\}] \cap f[\{y\}] \subseteq f[\{x\} \cap \{y\}] \]
        pero $l \in f[\{x\}] \cap f[\{y\}]$, entonces $l \in f[\{x\} \cap \{y\}]$, lo cual implica que $\{x\} \cap \{y\} \neq \emptyset$, por lo que $x=y$. Esto es un absurdo, ya que $x \in X \setminus Y$ y $y \ni Y$. El absurdo surge de suponer que $l \in f[Y]$, por lo tanto $l \notin f[Y]$, y asi $l \in f[X]\setminus f[Y]$, lo cual demuestra $f[X \setminus Y] \subseteq f[X] \setminus f[Y]$.
    \end{proof}

    \begin{ejercicio}{2.5}
        Sean $X,Y$ y $A$ conjuntos tales que $A \subseteq X$ y sea $f:X \to Y$ cualquier función. Definamos $i:A \to X $ para cada $a \in A$ como $i(a)=a$. Demuestra que para cualquier subconjunto $B \subseteq Y$ se da la igualdad $(f \circ i)^{-1}[B]=A \cap f^{-1}[B]$.
    \end{ejercicio}
    \begin{proof}
        Sea $B \subseteq Y$ cualquera, veamos que $(f \circ i)^{-1}[B]=A \cap f^{-1}[B]$.

        ($\subseteq$) Sea $x \in (f \circ i)^{-1}[B]$, entonces $\big( f \circ i \big)(x) \in B$. Por definición de composición y de $i$: $\big( f \circ i \big)(x) = f\big( i(x) \big)=f(x)$, así que $f(x) \in B$ y con ello $x \in f^{-1}[B]$. Además, como $\dom(f \circ i)=A$, entonces $(f \circ i)^{-1}[B] \subseteq A$, por lo que $x \in A$. Esto es, $x \in A \cap f^{-1}[B]$, por lo que:
        \[(f \circ i)^{-1}[B] \subseteq A \cap f^{-1}[B]\]

        ($\supseteq$) Sea $y \in A \cap f^{-1}[B]$ cualquiera. Entonces $y \in A$ y $y \in f^{-1}[B]$, de esto último $f(y) \in B$. Además, dado que $y \in A$, entonces $\big( f \circ i \big)(y) = f\big( i(y) \big)=f(y)$, por lo que $\big( f \circ i \big)(y) \in B$, mostrando que $y \in (f \circ i)^{-1}[B]$.
        \[A \cap f^{-1}[B] \subseteq (f \circ i)^{-1}[B]\]
    \end{proof}
    
\end{document}