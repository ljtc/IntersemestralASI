\documentclass[letterpaper,DIV=14,headsepline,12pt]{scrartcl}
%Utilidades en texto
\usepackage{amsthm}
\usepackage{thmtools}
\usepackage{array}
\usepackage{multicol, paracol}
%\usepackage{truthtable}
\usepackage[x11names, table]{xcolor}
\usepackage{arydshln}
\setlength{\dashlinedash}{0.5pt} % Longitud de los trazos
\setlength{\dashlinegap}{1.5pt}  % Espacio entre trazos
\usepackage[most]{tcolorbox}
%Stix2 como fuente
\usepackage[spanish,es-lcroman,mexico,shorthands=off]{babel}
\usepackage{stix2}
\usepackage{mathtools}
%Referencias
\usepackage[colorlinks=true, allcolors=VioletRed2]{hyperref}
\newcommand{\customRef}[2]{\hyperref[#1]{#2~\ref*{#1}}}
%Etiquetas de las enumeraciones
\usepackage[shortlabels]{enumitem}
\setenumerate[1]{label=\MakeLowercase{\roman*}), ref=\roman*}
\setenumerate[2]{label=\MakeLowercase{\arabic*}), ref=\alph*}
\usepackage{ifthen}

%Notación de conjuntos
\providecommand\st{\;|\;}
\providecommand\given{}
\newcommand\SetSymbol[1][]{\nonscript\:#1\vert\allowbreak\nonscript\:\mathopen{}}
\DeclarePairedDelimiterX\Set[1]\{\}{\renewcommand\given{\SetSymbol[\delimsize]}#1}


%Ejercicios
\newcounter{Ejer}
\newcounter{Pts}
\setcounter{Ejer}{1}
\setcounter{Pts}{1}
\newcommand{\pts}{}
\newenvironment{ejercicio}[1]{\noindent
    \ifthenelse{\equal{#1}{1} \OR \equal{#1}{+1}}{\renewcommand{\pts}{\textbf{(#1 pt)}}}{\renewcommand{\pts}{\textbf{(#1 pts)}}}\textbf{Ej. \theEjer} \pts\stepcounter{Ejer}}{\vspace{.3cm}}

%Marcador
    \newcommand{\marker}[2][Gold2]{
        \ifmmode
        \tcbox[on line,boxrule=0pt,colframe=#1!30,colback=#1!30,arc=2mm,left=0pt,right=0pt,top=.7pt,bottom=.7pt]{\ensuremath{#2}}%
        \else
        \tcbox[on line,boxrule=0pt,colframe=#1!30,colback=#1!30,arc=2mm,left=0pt,right=0pt,top=.7pt,bottom=.7pt]{#2}%
        \fi}

%Entorno de Demostración y Solución
\renewcommand\qedsymbol{$\blacksquare$}
\makeatletter
    \renewenvironment{proof}[1][]{%
        \par\pushQED{\qed}%
        \normalfont\topsep6pt \partopsep0pt % espacio antes del entorno
        \trivlist
        \item[\hskip\labelsep
                \textbf{\textit{Demostración.}}% Cambiado aquí
        ]#1
        }{%
        \popQED\endtrivlist\@endpefalse
    }
    \newenvironment{solu}[1][]{%
        \par\pushQED{\hfill \lozenge}%
        \normalfont\topsep6pt \partopsep0pt % espacio antes del entorno
        \trivlist
        \item[\hskip\labelsep
                \textbf{\textit{Solución.}}% Cambiado aquí
        ]#1
        }{%
        \popQED\endtrivlist\@endpefalse
    }
\makeatother

%Comandos que utilizamos
\newcommand{\id}{\mathrm{id}}
\newcommand{\op}{{}^{\mathrm{op}}}
\newcommand{\set}[1]{\{#1\}}
\renewcommand{\emptyset}{\varnothing}
\DeclareMathOperator{\ima}{ima}
\DeclareMathOperator{\dom}{dom}
\newcommand{\quot}[2]{{\raisebox{.2em}{$#1$}\left/\raisebox{-.2em}{$#2$}\right.}}

\pagenumbering{gobble}

\begin{document}
    \pagenumbering{gobble}

    \begin{center}
        {\fontsize{30}{60}\rmfamily \textbf{Tarea 3}} \\ \vspace{.2cm}
        Álgebra Superior 1, 2025-4
        
    \end{center}
    \begin{flushright}
        \footnotesize \hfill Profesor: Luis Jesús Trucio Cuevas.\\
        \hfill Ayudante: Hugo Víctor García Martínez.
    \end{flushright}
    
    \noindent\textit{\textbf{Instrucciones.} Resuelve los siguientes ejercicios. Esta tarea es individual y deberá ser entregada presencialmente, durante la clase del \textbf{viernes 4 de julio}.}\vspace{.4cm}

    \begin{ejercicio}{1}
        Se dice que una función $h:X \to Y$ es \textbf{constante} si y sólo si para cualesquiera $x,y \in X$ se tiene que $h(x)=h(y)$.
        \begin{enumerate}
            \item Demuestra que la composición de funciones constantes es una función constante.
            \item Encuentra dos funciones \textit{no} constantes; $f:\mathbb{N} \to \mathbb{Z}$ y $g:\mathbb{Z}\to \mathbb{N}$, cuya composición (la función $g \circ f:\mathbb{N} \to \mathbb{N}$) sea constante.
        \end{enumerate}
    \end{ejercicio}
    
    \begin{ejercicio}{1.5}
        En cada inciso, determina si la correspondiente función es inyectiva, sobreyectiva, o biyectiva. Demuestra la conclusión a la que llegaste (es decir, prueba si la función tiene o no la propiedad que se afrima).
        \begin{enumerate}
            \item $h:\{0,1,2\} \to \{x,y\}$ definida por $h=\{(0,x),(1,x),(2,y)\}$, aquí $x \neq y$.
            \item $A:\mathbb{R}\to \mathbb{R}$ dada para cada $x \in \mathbb{R}$ por $A(x)=4x+55$.
            \item $f:\mathbb{N}\to \mathscr{P}(\mathbb{N})$ dada para cada $n \in \mathbb{N}$ como $f(n)=\{0,n\}$.
        \end{enumerate} 
    \end{ejercicio}
    
    \begin{ejercicio}{2.5}
        Sean $A$ y $B$ conjuntos arbitrarios y $f:A \to B$ cualquier función. Demuestra que las siguientes proposiciones son equivalentes:
        \begin{enumerate}
            \item $\forall X \subseteq A \; \forall Y \subseteq A \; \big( f[X \setminus Y] \subseteq f[X] \setminus f[Y] \big)$.
            \item $\forall U \subseteq A \; \forall V \subseteq A \; \big( f[U] \cap f[V] \subseteq f[U \cap V] \big)$.
        \end{enumerate}
    \end{ejercicio}
    
    \begin{ejercicio}{2.5}
        Sean $X,Y$ y $A$ conjuntos tales que $A \subseteq X$ y sea $f:X \to Y$ cualquier función. Definamos $i:A \to X $ para cada $a \in A$ como $i(a)=a$. Demuestra que para cualquier subconjunto $B \subseteq Y$ se da la igualdad $(f \circ i)^{-1}[B]=A \cap f^{-1}[B]$.
    \end{ejercicio}

    \begin{ejercicio}{2.5}
        Sean $X$ un conjunto y $\phi:X \to X$ cualquier biyección. Demuestre que, para toda función $f:X \to X$ se da la equivalencia:
        \[ f \text{ es biyectiva } \quad \text{si y sólo si} \quad \, \phi^{-1} \circ f \circ \, \varphi \text{ es biyectiva}. \]
    \end{ejercicio}
    
\end{document}
%Para copiar en la tablet (Hugo): ^ `` ''