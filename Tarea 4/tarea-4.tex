\documentclass[letterpaper,DIV=14,headsepline,12pt]{scrartcl}
\usepackage[spanish,mexico,shorthands=off,es-noindentfirst]{babel}
\usepackage{stix2}
\usepackage{mathtools}
\usepackage{amsthm}
\usepackage{tikz-cd}
\tikzcdset{ font=\normalsize, every label/.append style = {font=\small} }
\usepackage{ifthen}
\usepackage[x11names, table]{xcolor}
\usepackage[most]{tcolorbox}
\usepackage[colorlinks=true, allcolors=VioletRed2]{hyperref}
\usepackage{newunicodechar}
\newunicodechar{¿}{?`}
\usepackage{float}
%\usepackage{multicol} \usepackage{paracol}
\usepackage[shortlabels]{enumitem}
\setenumerate[1]{label=\MakeLowercase{\roman*}), ref=\roman*}
\setenumerate[2]{label=\MakeLowercase{\arabic*}), ref=\alph*}

\usepackage{scrlayer-scrpage}
    \clearpairofpagestyles
    \ihead{\footnotesize \textit{Álgebra Superior I}} \ohead{\footnotesize
    \textit{Intersemestral 2025-4}}
    \cfoot{\normalfont\thepage}
    \addtokomafont{title}{\bfseries \rmfamily}
    \setlength{\headsep}{5pt}
    \newpairofpagestyles{beginstyle}{
        \clearpairofpagestyles
        \KOMAoptions{headsepline=false}
        \cfoot{\footnotesize \pagemark}
        \cfoot{\normalfont\thepage}
    }
    \addtokomafont{section}{\raggedleft \rmfamily}
    \addtokomafont{subsection}{\raggedleft\rmfamily}
    \setlength{\headsep}{7pt}
    \setlength{\footskip}{20pt}
    \setlength{\textheight}{550pt}

%Colors
\definecolor{azul}{RGB}{0, 60, 113}
\definecolor{dorado}{RGB}{196, 151, 57}
\definecolor{morado}{RGB}{101, 43, 145}
%Marcador
\newcommand{\marker}[2][Gold2]{
    \ifmmode
    \tcbox[on
    line,boxrule=0pt,colframe=#1!30,colback=#1!30,arc=2mm,left=0pt,right=0pt,top=.7pt,bottom=.7pt]{\ensuremath{#2}}%
    \else
    \tcbox[on
    line,boxrule=0pt,colframe=#1!30,colback=#1!30,arc=2mm,left=0pt,right=0pt,top=.7pt,bottom=.7pt]{#2}%
    \fi
}

%Cajas para teoremas y esas cosas
\usepackage{thmtools}
\usepackage[framemethod=TikZ]{mdframed}
\mdfsetup{innertopmargin=-2pt, innerbottommargin=5pt, roundcorner=5pt, linewidth=0}
\newmdtheoremenv[backgroundcolor=morado!10]{definicion}{Definición}
\newmdtheoremenv[backgroundcolor=dorado!10]{teorema}[definicion]{Teorema}
\newmdtheoremenv[backgroundcolor=gray!10]{ejj}{Ej}

\newcommand{\pts}{}
\newenvironment{ejercicio}[1]{\ifthenelse{\equal{#1}{1} \OR
\equal{#1}{+1}}{\renewcommand{\pts}{\textbf{(#1
pt)}}}{\renewcommand{\pts}{\textbf{(#1 pts)}}}\begin{ejj}\upshape
\pts}{\end{ejj}}

%Ejercicios

%\newenvironment{ejercicio}[1]{\vspace{.3cm}\noindent
    %\ifthenelse{\equal{#1}{1} \OR
    %\equal{#1}{+1}}{\renewcommand{\pts}{\textbf{(#1
    %pt)}}}{\renewcommand{\pts}{\textbf{(#1 pts)}}}\textbf{Ej. \theEjer}
    %\pts\stepcounter{Ejer}}{\vspace{.3cm}}

% Para escribir el "tal que" de los conjuntos
\providecommand\st{\;|\;} \providecommand\tq{\;|\;}

%Para el uso de \Set y \Set*
\providecommand\given{}
\newcommand\SetSymbol[1][]{\nonscript\:#1\vert\allowbreak\nonscript\:\mathopen{}}
\DeclarePairedDelimiterX\Set[1]\{\}{\renewcommand\given{\SetSymbol[\delimsize]}#1}
\DeclarePairedDelimiterX\Par[1](){#1}

%Comandos que utilizamos
\newcommand{\id}{\mathrm{id}} \newcommand{\op}{{}^{\mathrm{op}}}
\newcommand{\set}[1]{\{#1\}}
\renewcommand{\emptyset}{\varnothing}
\DeclareMathOperator{\ima}{ima} \DeclareMathOperator{\dom}{dom}
\newcommand{\quot}[2]{{\raisebox{.2em}{$#1$}\left/\raisebox{-.2em}{$#2$}\right.}}


%Corrección de las flechas de tikz-cd
\tikzcdset{arrow style=math font}
\tikzcdset{every arrow/.append style={line width=0.068em}}%ancho
\pgfmathdeclarefunction*{axis_height}{0}{%
    \begingroup%      
      \pgfmathreturn0.259em \endgroup}% 

\begin{document}

    \begin{center}
        {\fontsize{30}{60}\rmfamily \textbf{Tarea 4}} \\ \vspace{.2cm} Álgebra
        Superior 1, 2025-4
    \end{center}
    \begin{flushright}
        \footnotesize \hfill Profesor: Luis Jesús Trucio Cuevas.\\
        \hfill Ayudante: Hugo Víctor García Martínez.
    \end{flushright}

    \noindent\textit{\textbf{Instrucciones.} Resuelve los siguientes ejercicios.
    Esta tarea es individual y deberá ser entregada \textbf{presencialmente},
    durante la clase del \textbf{lunes 28 de julio.} \vspace{.4cm}}

    La idea de esta tarea es demostrar la versión para conjuntos de uno de los
    resultados más importantes en matemáticas; el Primer Teorema de Isomorfismo
    de Noether\footnote{Emmy Noether (1882-1935) fue una matemática alemana
    reconocida por sus contribuciones fundamentales en la física; y claro, en
    las matemáticas, particularlemten en álgebra abstracta. Existen muchos
    resultados que llevan su nombre en su honor, y el de esta tarea es la
    versión para conjuntos de uno de ellos.}. Existen versiones de este teorema
    para áreas como la Teoría de Grupos, la Teoría de Anillos, la Topología,
    etcétera; pero la formulación que se deemostrará en esta tarea es la
    siguiente:

    \begin{teorema}[Primero de Isomorfismo de Noehter]\label{teo:primero}
        Cualquier función $f\colon A \to B$ se puede escribir como composición
        de: una sobreyección, una biyección y una inyección. Es decir, existen
        conjuntos $X$ y $Y$, y funciones $s\colon A \to U$ sobreyectiva, $b\colon U \to V$
        biyectiva e $i\colon V \to B$ inyectiva, tales que $f = i \circ s \circ b$.
        \begin{equation}\label{eq:obj}
            \begin{tikzcd}
            A && B \\
            \\
            U && V
            \arrow["f", from=1-1, to=1-3]
            \arrow["s"', from=1-1, to=3-1]
            \arrow["b"', from=3-1, to=3-3]
            \arrow["i"', from=3-3, to=1-3]
        \end{tikzcd}
        \end{equation}
    \end{teorema}

    \textit{\textbf{Nota:} Desde ahora, y durante toda la tarea, los conjuntos
    $A$, $B$ y la función $f\colon A \to B$ serán fijos (pero arbitrarios).}

    \section*{Caso sencillo}

    Notemos que si $f\colon A \to B$ es una biyección, entonces el
    \autoref{teo:primero} es sencillo, pues
    podemos escribir a $f$ como $\id_B \circ f \circ \id_A$ (y las identidades
    de $A$ y $B$ son sobreyectiva e inyectiva, respectivamente). Por lo que el
    verdadero problema es sabér qué ocurre si $f$ no es sobreyectiva, o, si no
    es inyectiva.

    Como, por definición, un codominio para una función es cualquier conjunto
    que contenga a la imagen de la función, siempre se puede pensar que una
    función tiene por codominio a su imagen. En este sentido, $f\colon A \to B$ puede
    ser pensada con codominio $f[A]$; lo cual permite pensar a $f$ como una
    función sobreyectiva. A la función $f$ pensada así, se le denotará por
    $\hat{f}$, esto es:

    \begin{definicion}
        Se define $\hat{f}\colon A \to f[A]$ como la función $f$ pensada con codominio
        $f[A]$, es decir, para cada $x \in A$, $\hat{f}(x)=f(x)$.
    \end{definicion}

    Entonces, como conjuntos $f$ y $\hat{f}$ son iguales, pues:
    \begin{enumerate}
        \item $\dom(f)=\dom(\hat{f})=A$.
        \item Para cada $x \in A$, $f(x)=\hat{f}(x)$.
    \end{enumerate}
    
    La única diferencia es el codominio con el que las pensamos.

    \begin{ejercicio}{1}\label{ej:sobresuimagen} Demuestra que la función
        $\hat{f}\colon A \to f[A]$ siempre es función sobreyectiva; y concluye que, si
        $f\colon A \to B$ es inyectiva, entonces $\hat{f}\colon A \to f[A]$ es biyectiva.
    \end{ejercicio}

    Por lo tanto, la no sobreyectividad de $f$ no representa un obstáculo grande
    al momento de demostrar el \autoref{teo:primero}; esto es porque, $f\colon A \to
    B$ se puede pensar como primero aplicar $\hat{f}\colon A \to f[A]$ y luego una
    función que tome cada punto en $f[A]$ y, sin hacerele nada, lo piense como
    punto de $B$. Las funciones que se encargan de esto último se llaman
    ``inclusiones'', y su definición es la siguiente:

    \begin{definicion}
        Si $X$ y $Y$ son conjuntos tales que $X \subseteq Y$, definimos la
        \textit{inclusión} de $X$ en $Y$ como la función $i_{X,Y}\colon X \to Y$
        definida para cada $x \in X$ como $i_{X,Y}(x) = x$.

        \textit{Como conjunto, $i_{X,Y}=\id_X$, así que esta función no tendría
        que depender de $Y$; sin embargo, escribimos las inclusiones así para
        dejar en claro cuál es el codominio con el que las pensamos.}
    \end{definicion}

    \begin{ejercicio}{1.5}
        Demuestra que:
        \begin{enumerate}
            \item Si $X \subseteq Y$, entonces $i_{X,Y}\colon X \to Y$ es inyectiva.
            \item $f=i_{f[A],B} \circ \hat{f}$.
        \end{enumerate}
    \end{ejercicio}

    Por lo tanto, ya hemos resuelto una pqueña parte del problema que nos ataña,
    pues hemos demostrado que cualquier función $f\colon A \to B$ se puede escribir
    como composición de una sobreyección y una inyección, es decir, en términos
    del \hyperref[eq:obj]{Diagrama~\ref*{eq:obj}}, tenemos descubierto quiénes
    deberían ser $V$ e $i$:
    \begin{equation}\label{eq:hat}
        \begin{tikzcd}
            A && B \\
            \\
            \textcolor{Cornsilk4}{U} && {f[A]}
            \arrow["f", from=1-1, to=1-3]
            \arrow["s"', from=1-1, to=3-1, draw=Cornsilk4, text=Cornsilk4]
            \arrow["{\hat{f}}"{description}, from=1-1, to=3-3]
            \arrow["b"', from=3-1, to=3-3, draw=Cornsilk4, text=Cornsilk4]
            \arrow["{i_{f[A],B}}"', from=3-3, to=1-3]
        \end{tikzcd}
    \end{equation}

    Ahora falta lo más dificil, probar que $\hat{f}$ se puede descomponer como
    composición de una sobreyección, y luego, una biyección; es decir, falta
    descubrir quiénes son $s$, $b$ y $U$ en el
    \hyperref[eq:hat]{Diagrama~\ref*{eq:hat}}.

    \section*{``Inyectivizando'' una función}

    Una vez resuelto el problema de la sobreyectividad, el siguiente obstáculo
    es la inyectividad. ¿Cómo se puede transformar una función no inyectiva en
    otra que sí sea biyectiva? Imaginemos el siguiente caso particular:
    \begin{figure}[H]\label{fig:ejemplo}
        \begin{center}
            \begin{tikzpicture}
                \draw[DodgerBlue3, fill=DodgerBlue3!10, line width=2pt] (0,-.5)
                ellipse (0.8cm and 2.8cm); \draw[Chartreuse4,
                fill=Chartreuse4!10, line width=2pt] (5,-.5) ellipse (0.8cm and
                2cm);

                \node at (0,3) {$X=\{0,1,2,3,4\}$}; \node at (5,3)
                {$Y=\{a,b,c\}$};

                \node at (0,1.5) {$0$}; \node at (0,.5) {$1$}; \node at (0,-.5)
                {$2$}; \node at (0,-1.5) {$3$}; \node at (0,-2.5) {$4$};

                \node at (5,.5) {$a$}; \node at (5,-.5) {$b$}; \node at (5,-1.5)
                {$c$};

                \draw[-latex] (.2,1.5) -- (4.8,.5); \draw[-latex] (.2,.5) --
                (4.8,.5); \draw[-latex] (.2,-.5) -- (4.8,-1.5); \draw[-latex]
                (.2,-1.5) -- (4.8,-.5); \draw[-latex] (.2,-2.5) -- (4.8,-1.5);

                \draw[->, thick] (1.7,3) -- (3.8,3)  node[midway, above] {$g$};
            \end{tikzpicture}
        \end{center}
        \caption{Ejemplo particular de función}
    \end{figure}

    Aunque el dominio de $g$ conste de cinco puntos, se puede pensar que $g$
    aplica únicamente tres reglas: \textit{mandar hacia $a$, mandar hacia $b$} o
    \textit{mandar hacia $c$}, según sea el caso. Por lo cual, podemos
    clasificar a los elementos de $X$ en tres \textit{clases}: los que se mandan
    a $a$, los que se mandan a $b$ y los que se mandan a $c$ bajo $g$. Tal
    clasificación resulta en: $\{0,1\}=g^{-1}[\{a\}]$, $\{3\}=g^{-1}[\{b\}]$ y
    $\{2,4\}=g^{-1}[\{c\}]$, respectivamente.

    \begin{definicion}
        Definimos la relación $\sim$ en $A$ como: $ x \sim y \; \text{si y sólo
        si} \; f(x)=f(y)$
    \end{definicion}

    \begin{ejercicio}{1.5}\label{ej:rela} Por un ejercicio de las taras
        (\textit{¿cuál?}), la relación $\sim$ es de equivalencia. Demuestra la
        igualdad de conjuntos: $\quot{A}{\sim}=\{ f^{-1}[\{b\}] \mid b \in f[A]
        \}$.
    \end{ejercicio}

    Retomando el ejemplo particular de $g$ \hyperref[fig:ejemplo]{Figura~1}, se
    nota que:
    \begin{enumerate}
        \item Se puede mandar cada elemento $x\in X$ a su clase de equivalencia,
        por ejemplo $1$ a $\{0,1\}$.
        \item Cada clase surge de agrupar los elementos de $X$ según $g$ actúa
        en ellos. Luego, a la clase $\{0,1\}=g^{-1}[\{a\}]$ se le puede asignar
        el valor $a \in Y$ (pues esta clase se construyó como todos aquellos
        elementos de $X$ que al aplicarles $g$ daban $a$); a la calse
        $\{3\}=g^{-1}[\{b\}]$ se le puede asignar $b$ y a la clase
        $\{2,4\}=g^{-1}[\{c\}]$ se le puede asignar $c$.
    \end{enumerate}
    \begin{figure}[H]\label{fig:g}
        \begin{center}
            \begin{tikzpicture}
                \draw[DodgerBlue3, fill=DodgerBlue3!10, line width=2pt] (0,-.5)
                ellipse (0.8cm and 2.8cm); \draw[Goldenrod3, fill=Goldenrod3!10,
                line width=2pt] (5,-.5) ellipse (2.3cm and 2cm);
                \draw[Chartreuse4, fill=Chartreuse4!10, line width=2pt] (10,-.5)
                ellipse (0.8cm and 2cm);

                \node at (0,3) {$X=\{0,1,2,3,4\}$}; \node at (5,-3)
                {$\quot{X}{\sim}$}; \node at (10,3) {$Y=\{a,b,c\}$};

                \node at (0,1.5) {$0$}; \node at (0,.5) {$1$}; \node at (0,-.5)
                {$2$}; \node at (0,-1.5) {$3$}; \node at (0,-2.5) {$4$};

                \node at (5,.5) {$\{0,1\}=g^{-1}[\{a\}]$}; \node at (5,-.5)
                {$\{3\}=g^{-1}[\{b\}]$}; \node at (5,-1.5)
                {$\{2,4\}=g^{-1}[\{c\}]$};

                \node at (10,.5) {$a$}; \node at (10,-.5) {$b$}; \node at
                (10,-1.5) {$c$};

                \draw[-latex] (.2,1.5) -- (3.4,.5); \draw[-latex] (.2,.5) --
                (3.4,.5); \draw[-latex] (.2,-1.5) -- (3.55,-.5); \draw[-latex]
                (.2,-.5) -- (3.4,-1.5); \draw[-latex] (.2,-2.5) -- (3.4,-1.5);

                \draw[-latex] (6.6,.5) -- (9.8,.5); \draw[-latex] (6.4,-.5) --
                (9.8,-.5); \draw[-latex] (6.6,-1.5) -- (9.8,-1.5);

                \draw[->, thick] (1.7,3) -- (8.8,3)  node[midway, above] {$g$};
            \end{tikzpicture}
        \end{center}
        \caption{Descomposición de $g$ en dos procesos (o funciones).}
    \end{figure}

    La imagen anterior describe dos procesos (o funciones) cuya aplicación
    seguida resulta en hacer exactamente lo mismo que $g$, los estudiaremos con
    detenimiento a través de los siguientes ejercicios. El primero de ellos se
    define así:

    \begin{definicion}
        Definimos $s_f\colon A \to \quot{A}{\sim}$ por medio de: $s_f(x)=[x]$.
    \end{definicion}

    \begin{ejercicio}{1.5}
        En términos de la definición previa:
        \begin{enumerate}
            \item Demuestra que $s_f$ es sobreyectiva.
            \item Prueba que si $f$ es inyectiva, entonces para cada $x \in X$,
            $[x]=\{x\}$.
            \item Concluye que, si $f$ es inyectiva, entonces $q_f$ es
            biyectiva.
        \end{enumerate}
    \end{ejercicio}

    \begin{ejercicio}{1.5}
        Haciendo uso de la terminología definida hasta ahora, para cada inciso
        da un ejemplo particular de:
        \begin{enumerate}
            \item Una función $f\colon \{1,2,3,4,5,6,7,8\} \to \{0,1,2\}$ tal que
            $s_f$ sea constante.
            \item Una función $f\colon \mathbb{N} \to \mathbb{N}$ de modo que
            $\quot{\mathbb{N}}{\sim}$ posea exactamente $5$ elementos.
            \item Una función $f\colon \mathbb{Z} \to \mathbb{Z}$ tal que para cada $z
            \in \mathbb{Z}$ se cumple $s_f(z) \cap s_f(z+7)$.
        \end{enumerate}
    \end{ejercicio}

    Con lo hecho hasta el momento, se sugiere que el conjunto $U$ del
    \hyperref[eq:hat]{Diagrama~\ref*{eq:hat}} debe ser el cociente
    $\quot{A}{\sim}$, y, la función $s$ debe ser $s_f$; resultando en:
    \begin{equation}\label{eq:casi}
        \begin{tikzcd}
            A && B \\
            \\
            \quot{A}{\sim} && {f[A]}
            \arrow["f", from=1-1, to=1-3]
            \arrow["s_f"', from=1-1, to=3-1]
            \arrow["{\hat{f}}"{description}, from=1-1, to=3-3]
            \arrow["b"', from=3-1, to=3-3, draw=Cornsilk4, text=Cornsilk4]
            \arrow["{i_{f[A],B}}"', from=3-3, to=1-3]
        \end{tikzcd}
    \end{equation}

    Únicamente resta definir $b\colon \quot{A}{\sim} \to f[A]$ biyectiva, de modo que
    se tenga la composición que queremos, esto es justamente lo que representa
    la segunda función aplicada en la \hyperref[fig:g]{Figura~2} del ejemplo
    particular. Éste puede ser pensado como asignarle a cada clase
    $g^{-1}[\{y\}]$ el elemento $y$. Por supuesto, hay que verificar que tal
    cosa es función, es decir, está bien definida. Así que a continuación la
    definiremos únicamente como relación; y, posteriormente se verificará que es
    función:
    \begin{definicion}
        Definimos $f^* \subseteq \quot{A}{\sim} \times f[A]$ como la relación:
        \[  M f^* b \quad \text{si y sólo si} \quad \exists y \in M \big( b=f(y)
        \big) \]
    \end{definicion}

    Nada (aún) garantiza que $f^*$ sea función, es decir, podría existir cierto
    $M \in \quot{A}{\sim}$ que se relacioine con dos elementos diferentes $b,b'
    \in f[A]$, es decir $M f^* b$ y $M f^* b'$, lo cual prohibiría que $f^*$ sea
    función.
    
    Lo anterior no debería ocurrir, notemos que si $M \in \quot{A}{\sim}$
    entonces $M=[x]$ para cierto $x \in A$. Luego, si $M f^* b$, es decir, $[x]
    f^* b$; por definición de $f^*$, existe $y \in [x]$ de modo que $b=f(y)$;
    simimilarmente, $b'=f(y')$ para algún $y'\in[x]$, a partir de aquí ¿Cómo se
    tiene la garantía de que $b=b'$? \dots

    \begin{ejercicio}{1.5}\label{ej:fStar} Completa la demostración empezada en
        el párrafo anterior; es decir, prueba que $f^*$ es función de
        $\quot{A}{\sim}$ en $f[A]$ mostrando que:
        \[ \forall M \in \quot{A}{\sim} \; \forall b,b' \in f[A] \; \Big( \big(
        M f^* b \land M f^* b' \big) \to b=b' \Big) \]
    \end{ejercicio}

    \section*{Juntando todo}

    Es bueno (y necesario) meditar que, lo que demuestra el ejercicio anterior
    en el fondo es que, para cada $x \in A$:
    \begin{equation}\label{eq:fStar}
        f^*([x])=f(x)
    \end{equation}

    \textit{Notemos que $[x]$; al ser una clase de equivalencia, podría ``tener
    otro nombre'', como $[u]$ o $[z]$ con $u,z \neq x$, lo cual imposibilita
    definir al inicio $f^*$ como en (\ref{eq:fStar}), esto es algo que se debía
    demostrar.}

    Se concluye esta tarea juntanto todo lo que tenemos, esto es, mostrando que
    hacer $f\colon A\to B$ es lo mismo que primero hacer $s_f\colon A\to \quot{A}{\sim}$,
    luego $f^*\colon \quot{A}{\sim} \to f[A]$, y por últimio, $i_{f[A],B}$. Esto es,
    se compelta el \hyperref[eq:obj]{Diagrama~\ref*{eq:casi}} descubriendo
    quiénes son exactamente todos los objetos que no conocíamos al inicioi, en
    el \hyperref[eq:obj]{Diagrama~\ref*{eq:obj}}.
    \begin{equation}\label{eq:ya}
        \begin{tikzcd}
            A && B \\
            \\
            \quot{A}{\sim} && {f[A]}
            \arrow["f", from=1-1, to=1-3]
            \arrow["s_f"', from=1-1, to=3-1]
            %\arrow["{\hat{f}}"{description}, from=1-1, to=3-3]
            \arrow["f^*"', from=3-1, to=3-3]
            \arrow["{i_{f[A],B}}"', from=3-3, to=1-3]
        \end{tikzcd}
    \end{equation}

    Lo único que falta probar es que $f^*$ es biyectiva, para esto, en lugar de
    usar la definición cruda de $f^*$, se sugiere utilizar la igualdad de la
    regla de correspondencia para $f^*$ dada en (\ref{eq:fStar}).

    \begin{ejercicio}{1}
        Demuestra que $f^*\colon \quot{A}{\sim} \to f[A]$ es biyectiva.
    \end{ejercicio}

    Y finalmente:
    \begin{ejercicio}{.5}
        Demuestra el Primer Teorema de Isomorfismo para conjuntos; es decir,
        prueba que:
        \[ f=i_{f[A],B} \; \circ \; f^* \; \circ \; s_f \] ($f$ es composición
        de una sobreyección, una biyección y una inyección).
    \end{ejercicio}

    \section*{Por puntos extra\dots}

    Tras realizar todo lo que acabamos de hacer, quedan varias dudas que sólo
    los más avispados podrían plantearse. La primera de ellas es la siguiente:
    ¿La notación $(-)^*$ es buena? es decir, ¿para cada $f\colon A \to B$, $f^*$ es la
    única función biyectiva $b\colon \quot{A}{\sim}$ que cumple $f=i_{f[A],B} \circ b
    \circ s_f$?, en efecto:
    
    \begin{ejercicio}{+2}
        Prueba que, si $j,k\colon \quot{A}{\sim} \to f[A]$ son funciones biyectivas
        tales que se dan las igualdades:
        \begin{align*}
            f & = i_{f[A],B} \; \circ \; j \; \circ \; s_f \\
            f & = i_{f[A],B} \; \circ \; k \; \circ \; s_f
        \end{align*}
        entonces $j=k$ (y por tanto ambas deben ser $f^*$).

        \textit{Hint: Recuerde que una función es sobreyectiva si y sólo si tiene\dots;
        y, es inyectiva si y sólo si tiene\dots}
    \end{ejercicio}
\end{document}