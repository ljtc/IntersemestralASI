\documentclass[preview]{standalone}
\usepackage{stix2}
\usepackage[x11names, table]{xcolor}

\begin{document}

Ejemplos de colores simples:

\textcolor{DodgerBlue3}{Este es un texto en color: \textbf{DodgerBlue3}.}

\textcolor{Tomato1}{Este es un texto en color: \textbf{Tomato1}.}

\textcolor{DarkOliveGreen4}{Este es un texto en color: \textbf{DarkOliveGreen4}.}

\textcolor{IndianRed1}{Este es un texto en color: \textbf{IndianRed1}.}

\textcolor{MediumOrchid3}{Este es un texto en color: \textbf{MediumOrchid3}.}

\textcolor{SlateBlue2}{Este es un texto en color: \textbf{SlateBlue2}.}

\textcolor{DeepPink3}{Este es un texto en color: \textbf{DeepPink3}.}

\textcolor{Goldenrod3}{Este es un texto en color: \textbf{Goldenrod3}.}

\textcolor{CadetBlue3}{Este es un texto en color: \textbf{CadetBlue3}.}

\textcolor{OrangeRed2}{Este es un texto en color: \textbf{OrangeRed2}.} \vspace{.5cm}

Ejemplos de colores con opacidad del 40\%:

\textcolor{DodgerBlue3!40}{Este es un texto en color: \textbf{DodgerBlue3}.}

\textcolor{Tomato1!40}{Este es un texto en color: \textbf{Tomato1}.}

\textcolor{DarkOliveGreen4!40}{Este es un texto en color: \textbf{DarkOliveGreen4}.}

\textcolor{IndianRed1!40}{Este es un texto en color: \textbf{IndianRed1}.}

\textcolor{MediumOrchid3!40}{Este es un texto en color: \textbf{MediumOrchid3}.}

\textcolor{SlateBlue2!40}{Este es un texto en color: \textbf{SlateBlue2}.}

\textcolor{DeepPink3!40}{Este es un texto en color: \textbf{DeepPink3}.}

\textcolor{Goldenrod3!40}{Este es un texto en color: \textbf{Goldenrod3}.}

\textcolor{CadetBlue3!40}{Este es un texto en color: \textbf{CadetBlue3}.}

\textcolor{OrangeRed2!40}{Este es un texto en color: \textbf{OrangeRed2}.}

\end{document}