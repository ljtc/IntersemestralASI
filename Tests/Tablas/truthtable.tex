\documentclass[letterpaper,DIV=14,headsepline,12pt]{scrartcl}
\usepackage[spanish,mexico,shorthands=off,es-lcroman]{babel}
\usepackage{stix2}
\pagenumbering{gobble}
%\setlength{\parindent}{0cm}

\usepackage{mathtools}
\usepackage{amsthm}
\usepackage{tikz-cd}
\usepackage{comment}
\usepackage{lipsum}
\usepackage{ifthen}
\usepackage{truthtable} %para las tablas de verdad

\usepackage{multicol}
\usepackage{paracol}
\usepackage[shortlabels]{enumitem}
\setenumerate[1]{label=\MakeLowercase{\roman*}), ref=\roman*}
\setenumerate[2]{label=\MakeLowercase{\arabic*}), ref=\alph*}

% Para escribir el "tal que" de los conjuntos
\providecommand\st{\;|\;}

%Para el uso de \Set y \Set*
\providecommand\given{}
\newcommand\SetSymbol[1][]{\nonscript\:#1\vert\allowbreak\nonscript\:\mathopen{}}
\DeclarePairedDelimiterX\Set[1]\{\}{\renewcommand\given{\SetSymbol[\delimsize]}#1}
\DeclarePairedDelimiterX\Par[1](){#1}
%Ejercicios
%\newcommand{\pts}[1]{%
  %\ifthenelse{\equal{#1}{1}}{\hfill \textbf{(#1 pt)}}{\hfill\textbf{(#1 pts)}}
%}

\newcounter{Ejer}
\newcounter{Pts}
\setcounter{Ejer}{1}
\setcounter{Pts}{1}
\newcommand{\pts}{}
\newenvironment{ejercicio}[1]{\noindent
    \ifthenelse{\equal{#1}{1}}{\renewcommand{\pts}{\textbf{(#1 pt)}}}{\renewcommand{\pts}{\textbf{(#1 pts)}}}\textbf{Ej. \theEjer} \pts\stepcounter{Ejer}}{\vspace{.3cm}}

%Comandos que utilizamos
\newcommand{\id}{\mathrm{id}}
\newcommand{\op}{{}^{\mathrm{op}}}
\newcommand{\set}[1]{\{#1\}}
\renewcommand{\emptyset}{\varnothing}
\DeclareMathOperator{\ima}{ima}
\DeclareMathOperator{\dom}{dom}

\begin{document}

    Esto es una prueba del comando \texttt{truthtable}.

    \begin{center}        
        \begin{tabular}{c|c||c|c|c|c|c|c|c}
            \truthtable{A,B}{$A$,$B$} %{variables sintácticas}{variables a imprimir}
            {!A, A & B, A | B, ^(A; B), !&(A; B), >>(A; B), <>(A; B)} %sintaxis
            {$ \lnot A$, $A \land B$, $A \lor B$,$A \veebar B$,$A | B$, $A \rightarrow B$, $A \leftrightarrow B$} %etiquetas
            {$V$}{$F$} %{etiqueta verdadero}{etiqueta falso}
        \end{tabular}
    \end{center}

    Con los símbolos clásicos:

    \begin{center}        
        \begin{tabular}{c|c||c|c|c|c|c}
            \truthtable{A,B}{$A$,$B$}
            {!A, A & B, A | B, >>(A; B), <>(A; B)}
            {$ \lnot A$, $A \land B$, $A \lor B$, $A \rightarrow B$, $A \leftrightarrow B$}
            {$V$}{$F$}
        \end{tabular}
    \end{center}

    Aquí un ejemplo diferente utilizando tres letras proposicionales:
    \begin{center}        
        \begin{tabular}{c|c|c||c|c|c}
            \truthtable{A,B,C}{$A$,$B$,$C$}
            { >>(A; >>(B; C)), >>( (A & B) ; C ), <>((>>(A; >>(B; C)));(>>( (A & B) ; C ))) }
            {$ A \to (B \to C) $, $(A \land B) \to C$, $(A \to (B \to C)) \leftrightarrow ((A \land B) \to C)$}
            {$1$}{$0$}
        \end{tabular}
    \end{center}
    
\end{document}