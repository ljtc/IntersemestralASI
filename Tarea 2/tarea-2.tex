\documentclass[letterpaper,DIV=14,headsepline,12pt]{scrartcl}
\usepackage[spanish,mexico,shorthands=off,es-lcroman]{babel}
\usepackage{stix2}
\pagenumbering{gobble}
%\setlength{\parindent}{0cm}

\usepackage{mathtools}
\usepackage{amsthm}
\usepackage{tikz-cd}
\usepackage{comment}
\usepackage{lipsum}
\usepackage{ifthen}

\usepackage{multicol}
\usepackage[shortlabels]{enumitem}
\setenumerate[1]{label=\MakeLowercase{\roman*}), ref=\roman*}
\setenumerate[2]{label=\MakeLowercase{\arabic*}), ref=\alph*}

% Para escribir el "tal que" de los conjuntos
\providecommand\st{\;|\;}

%Para el uso de \Set y \Set*
\newcommand\SetSymbol[1][]{%
    \nonscript\:#1\vert
    \allowbreak
    \nonscript\:
    \mathopen{}}
    \DeclarePairedDelimiterX\Set[1]\{\}{%
    \renewcommand\st{\SetSymbol[\delimsize]}
    #1
}
    \DeclarePairedDelimiterX\Class[1]\langle\rangle{%
    \renewcommand\st{\SetSymbol[\delimsize]}
    #1
}

%Ejercicios
%\newcommand{\pts}[1]{%
  %\ifthenelse{\equal{#1}{1}}{\hfill \textbf{(#1 pt)}}{\hfill\textbf{(#1 pts)}}
%}

\newcounter{Ejer}
\newcounter{Pts}
\setcounter{Ejer}{1}
\setcounter{Pts}{1}
\newcommand{\pts}{}
\newenvironment{ejercicio}[1]{\noindent\textbf{Ejercicio \theEjer.} \ifthenelse{\equal{#1}{1}}{\renewcommand{\pts}{\hfill \textbf{(#1 pt)}}}{\renewcommand{\pts}{\hfill\textbf{(#1 pts)}}}\stepcounter{Ejer}}{\pts \vspace{.3cm}}

%Comandos que utilizamos
\newcommand{\id}{\mathrm{id}}
\newcommand{\op}{{}^{\mathrm{op}}}
\newcommand{\set}[1]{\{#1\}}
\renewcommand{\emptyset}{\varnothing}
\DeclareMathOperator{\ima}{ima}
\DeclareMathOperator{\dom}{dom}

\begin{document}

    \begin{center}
        {\fontsize{30}{60}\rmfamily \textbf{Tarea 2}} \\ \vspace{.2cm}
        Álgebra Superior 1, 2025-4
    \end{center}
    \begin{flushright}
        \footnotesize \hfill Profesor: Luis Jesús Trucio Cuevas.\\
        \hfill Ayudante: Hugo Víctor García Martínez.
    \end{flushright}

    \noindent\textit{\textbf{Instrucciones.} Resuelve los siguientes ejercicios. Esta tarea es individual y deberá ser entregada presencialmente, durante la clase del \textbf{martes 24 de junio}.}\vspace{.4cm}

    \begin{ejercicio}{2}

    \end{ejercicio}

    \begin{ejercicio}{2}
        
    \end{ejercicio}

    \begin{ejercicio}{3}
        
    \end{ejercicio}

\end{document}