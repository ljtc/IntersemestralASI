\documentclass[letterpaper,DIV=14,headsepline,12pt]{scrartcl}
\usepackage[spanish,mexico,shorthands=off,es-lcroman]{babel}

%Paquetes
\usepackage{stix2}
\usepackage{mathtools}
\usepackage{amsmath,amsthm,amsfonts,amscd,amsbsy,amsxtra}
\usepackage{tikz-cd}
\usepackage{comment}
\usepackage{lipsum}
\usepackage{ifthen}
\usepackage{multicol}
\usepackage{paracol}

%Enumeraciones
\usepackage[shortlabels]{enumitem}
\setenumerate[1]{label=\MakeLowercase{\roman*}), ref=\roman*}
\setenumerate[2]{label=\MakeLowercase{\arabic*}), ref=\alph*}

% Para escribir el "tal que" de los conjuntos
\providecommand\st{\;|\;}

%Para el uso de \Set y \Set*
\providecommand\given{}
\newcommand\SetSymbol[1][]{\nonscript\:#1\vert\allowbreak\nonscript\:\mathopen{}}
\DeclarePairedDelimiterX\Set[1]\{\}{\renewcommand\given{\SetSymbol[\delimsize]}#1}
\DeclarePairedDelimiterX\Par[1](){#1}

%Ejercicios
\newcounter{Ejer}
\newcounter{Pts}
\setcounter{Ejer}{1}
\setcounter{Pts}{1}
\newcommand{\pts}{}
\newenvironment{ejercicio}[1]{\noindent
    \ifthenelse{\equal{#1}{1}}{\renewcommand{\pts}{\textbf{(#1 pt)}}}{\renewcommand{\pts}{\textbf{(#1 pts)}}}\textbf{Ej. \theEjer} \pts\stepcounter{Ejer}}{\vspace{.3cm}}

%Comandos que utilizamos
\newcommand{\id}{\mathrm{id}}
\newcommand{\op}{{}^{\mathrm{op}}}
\newcommand{\set}[1]{\{#1\}}
\renewcommand{\emptyset}{\varnothing}
\DeclareMathOperator{\ima}{ima}
\DeclareMathOperator{\dom}{dom}

\begin{document}
    \pagenumbering{gobble}

    \begin{center}
        {\fontsize{30}{60}\rmfamily \textbf{Tarea 2}} \\ \vspace{.2cm}
        Álgebra Superior 1, 2025-4
    \end{center}
    \begin{flushright}
        \footnotesize \hfill Profesor: Luis Jesús Trucio Cuevas.\\
        \hfill Ayudante: Hugo Víctor García Martínez.
    \end{flushright}

    \noindent\textit{\textbf{Instrucciones.} Resuelve los siguientes ejercicios. Esta tarea es individual y deberá ser entregada presencialmente, durante la clase del \textbf{martes 24 de junio}.}\vspace{.4cm}

    \begin{ejercicio}{1}
        Sean $I,J,K$ conjuntos no vacíos y supongamos que $J \cup K = I$. Si $\{X_i \st i \in I\}$ es una familia indexada de conjuntos, demuestra que:
        \[ \bigcap_{i \in I} X_i = \Big( \bigcap_{i \in J} X_i \Big) \cap \Big( \bigcap_{i \in K} X_i \Big) \]
    \end{ejercicio}

    \begin{ejercicio}{1}
        Sean $A,B,X$ y $Y$ conjuntos no vacíos. Demuestra:
        \begin{enumerate}
            \item $A \times B \subseteq X \times Y$ si y sólo si $A \subseteq X$ y $B \subseteq Y$.
            \item $A \times B = X \times Y$ si y sólo si $A=X$ y $B=Y$.
            \item $(A \setminus X) \times B = (A \times B) \setminus (X \times B) $.
        \end{enumerate}
        \textit{Sugerencia: Para (ii), utiliza el inciso (i) y el hecho de que dos conjuntos son iguales si y sólo si, uno está contenido en el otro}.
    \end{ejercicio}

    \begin{ejercicio}{3}
        Sean $A$ un conjunto y $R,S \subseteq A \times A$ relaciones sobre $A$. Demuestra que:
        \begin{enumerate}
            \item $R \cap S$ es reflexiva si y solamente si $R$ y $S$ son reflexivas.
            \item $R$ es simétrica si y sólo si $R=R^{-1}$.
        \end{enumerate}
    \end{ejercicio}

    \begin{ejercicio}{1}
        Sea $R$ una relación cualquiera. Prueba que, si $\dom(R) \cap \ima(R) = \emptyset$, entonces $R$ es antisimétrica. ¿Qué ocurre con el recíproco de lo anterior?, es decir, ¿Si $R$ es antisimétrica, entonces $\dom(R) \cap \ima(R) = \emptyset$?
    \end{ejercicio}

    \begin{ejercicio}{1}
        En cada inciso $R$ es una relación sobre un conjunto $A$. Indica en cada caso, si $R$ es: reflexiva, simétrica, transitiva, antireflexiva o antisimétrica. Si en algún caso $R$ es relación de orden parcial, o de equivalencia, indícalo. No es necesario justificar.
        \begin{enumerate}
            \item $A$ es el conjunto $\{\text{Piedra, Papel, Tijeras}\}$ y $R \subseteq A \times A$ la relación:
            \[ R := \Set*{(\text{Piedra}, \text{Tijeras}), (\text{Tijeras}, \text{Papel}), (\text{Papel}, \text{Piedra})} \]

            \item $A$ es el conjunto de todas las posibles rectas en el plano (digamos, $\mathbb{R}^2$) y $R \subseteq A \times A$ es la relación $R:=\{(x,y) \in A \times A \st x \text{ es paralela a } y\}$.
            
            \item $A=\mathbb{Z}$ y $R \subseteq A \times A$ está dada por $n \mathrel{R} m$ si y sólo si $n ^2 \leq m^2$.

            \item 
        \end{enumerate}
    \end{ejercicio}

\end{document}